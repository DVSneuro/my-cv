%------------------------------------
% Dario Taraborelli
% Typesetting your academic CV in LaTeX
%
% URL: http://nitens.org/taraborelli/cvtex
% DISCLAIMER: This template is provided for free and without any guarantee 
% that it will correctly compile on your system if you have a non-standard  
% configuration.
% Some rights reserved: http://creativecommons.org/licenses/by-sa/3.0/
%
% Adapted from https://github.com/dartar/cvtex
% by David V. Smith (david.v.smith@temple.edu)
%------------------------------------

%!TEX TS-program = xelatex
%!TEX encoding = UTF-8 Unicode

\documentclass[11pt, letterpaper]{article}
\usepackage{xunicode,fontspec,xltxtra} 
\usepackage{libertine} 

% DOCUMENT LAYOUT
\usepackage{geometry} 
\geometry{letterpaper, textwidth=5.5in, textheight=8.5in, marginparsep=7pt, marginparwidth=.6in}
\setlength\parindent{0in}

% FONTS
\usepackage[usenames,dvipsnames]{color}
\usepackage{xunicode}
\usepackage{xltxtra}
\defaultfontfeatures{Mapping=tex-text}
%\setromanfont [Ligatures={Common}, Numbers={OldStyle}, Variant=01]{Linux Libertine O}
%\setmonofont[Scale=0.8]{Monaco}

% ---- CUSTOM COMMANDS
\chardef\&="E050
\newcommand{\html}[1]{\href{#1}{\scriptsize\textsc{[html]}}}
\newcommand{\pdf}[1]{\href{#1}{\scriptsize\textsc{[pdf]}}}
\newcommand{\doi}[1]{\href{#1}{\scriptsize\textsc{[doi]}}}
\newcommand{\psyarxiv}[1]{\href{#1}{\textit{PsyArXiv}}}
\newcommand{\biorxiv}[1]{\href{#1}{\textit{bioRxiv}}}

% ---- MARGIN YEARS
\usepackage{marginnote}
\newcommand{\amper{}}{\chardef\amper="E0BD }
\newcommand{\years}[1]{\marginnote{\scriptsize #1}}
\renewcommand*{\raggedleftmarginnote}{}
\setlength{\marginparsep}{7pt}
\reversemarginpar

% HEADINGS
\usepackage{sectsty} 
\usepackage[normalem]{ulem} 
\sectionfont{\mdseries\upshape\LARGE}
\subsectionfont{\mdseries\scshape\large} 
\subsubsectionfont{\mdseries\upshape\large} 

\sectionfont{\sectionrule{0pt}{0pt}{-5pt}{0.8pt}}

%\usepackage{titlesec}
%
%\titleformat{\section}
 % {\normalfont\Large\bfseries}{\thesection}{1em}{}[{\titlerule[0.8pt]}]

%\usepackage{titlesec}
%\titleformat{\section}
% {\normalfont\normalsize\bfseries}{\thesection.}{1em}{}
%\titleformat{\subsection}
%  {\normalfont\normalsize\itshape}{\thesubsection.}{1em}{}
%\titleformat{\subsubsection}
%  {\normalfont\normalsize\itshape}{\thesubsubsection.}{1em}{}

% PDF SETUP
% ---- FILL IN HERE THE DOC TITLE AND AUTHOR
\usepackage[driverfallback=dvipdfm, bookmarks, colorlinks, breaklinks, 
% ---- FILL IN HERE THE TITLE AND AUTHOR
	pdftitle={David V. Smith - vita},
	pdfauthor={David V. Smith},
	pdfproducer={http://nitens.org/taraborelli/cvtex}
]{hyperref}  
\hypersetup{linkcolor=blue,citecolor=blue,filecolor=blue,urlcolor=blue} 


% IMPORT OTHER PACKAGES, INCLUDING PYTHONTEX
\usepackage{amsmath,amssymb}
\usepackage{fullpage}
\usepackage{graphicx}
\usepackage[svgnames]{xcolor}
\usepackage{url}
\urlstyle{same}
\usepackage{academicons}
\usepackage{hanging}
\usepackage{multicol}
\usepackage{enumitem}


%\defaultfontfeatures{Path = /usr/local/texlive/2017/texmf-dist/fonts/opentype/public/fontawesome/ }
\defaultfontfeatures{Extension = .otf}
\usepackage{fontawesome}

\usepackage[makestderr]{pythontex}
\restartpythontexsession{\thesection}
\usepackage[framemethod=TikZ]{mdframed}
\newcommand{\pytex}{Python\TeX}
%\renewcommand*{\thefootnote}{\fnsymbol{footnote}}


%\usepackage[dvipsnames]{xcolor}
%\definecolor{myblue}{RGB}{0, 26, 87}
%\definecolor{mygray}{gray}{0.6}


% START DOCUMENT
\begin{document}
\textbf{{\Huge David Victor Smith}}\\[.5cm]
\begin{tabbing}
Assistant Professor \hspace{1.75in} \= \aiGoogleScholar \hspace{.2cm} \= \href{https://scholar.google.com/citations?user=czV7OcQAAAAJ&hl=en}{scholar.google.com} \\ 
Department of Psychology \> \aiResearchGate \> \href{https://www.researchgate.net/profile/David_Smith96/}{researchgate.net} \\
Temple University \> \faGithub \>  \href{https://github.com/DVS-Lab/}{github.com/DVS-Lab} \\
825 Weiss Hall \> \aiOrcid \> \href{https://orcid.org/0000-0001-5754-9633}{orcid.org/0000-0001-5754-9633} \\
1701 North 13th Street \>  \aiPublons \> \href{https://publons.com/author/1204254/david-v-smith}{publons.com} \\
Philadelphia, PA, 19122 \> \aiOSF \> \href{https://osf.io/5zq6h/}{osf.io} \\[.2cm]
phone: 215-204-1552  \> \faTwitter \> \href{https://twitter.com/DVSneuro}{twitter.com/DVSneuro} \\
email: \href{mailto:david.v.smith@temple.edu}{david.v.smith@temple.edu} \> \faUniversity \> \href{https://sites.temple.edu/neuroeconlab/}{sites.temple.edu/neuroeconlab} \\ 
 \end{tabbing}

%\faInstitution
%\faLinux
%\faLink
%\faUniversity
 
%https://osf.io/5zq6h/
%https://twitter.com/DVSneuro
%https://github.com/DVS-Lab
%https://sites.temple.edu/neuroeconlab/
%https://orcid.org/0000-0001-5754-9633
%https://publons.com/author/1204254/david-v-smith
%http://www.researcherid.com/rid/A-5487-2011
%https://www.researchgate.net/profile/David_Smith96


%%\hrule \section*{Professional Experience}
\section*{Professional Experience}
\begin{tabbing}
\years{2017-present} \textsc{Temple University} \hspace{.75in} \= \textbf{Assistant Professor}, Department of Psychology \\[.2cm]
\years{2012-2016} \textsc{Rutgers University} \> \textbf{Postdoctoral Fellow}, Department of Psychology \\[.2cm]
\years{2006-2012} \textsc{Duke University} \> \textbf{Graduate Student}, Center for Cognitive Neuroscience \\
\end{tabbing}


%\hrule \section*{Education \& Training}
\section*{Education and Training}
\begin{tabbing}
\years{2012-2016}\textsc{PostDoc} \hspace{.75in} \= \textsc{Rutgers University} \\
\> Department of Psychology \\
\> \textsc{Advisor}: Dr. Mauricio Delgado \\ [.2cm]

\years{2012}\textsc{PhD} \> \textsc{Duke University} \\
\> Department of Psychology and Neuroscience \\
\> \textsc{Advisor}: Dr. Scott Huettel \\ [.2cm]

\years{2006}\textsc{BS} \> \textsc{University of South Carolina} \\
\> Experimental Psychology, \textit{Magna Cum Laude} \\
\> \textsc{Advisors}: Drs. Gordon Baylis \& Chris Rorden \\
 \end{tabbing}


%\hrule \section*{Honors \& Awards}
\section*{Honors and Awards}
\noindent
\years{2019}Excellence in Mentoring, Temple University, Department of Psychology Honors Program\\
\years{2019}Faculty Fellow, Public Policy Lab, Temple University\\
\years{2016}Rising Star, Association for Psychological Science\\
\years{2016}Young Investigator Travel Award, NIDA Symposium on Persistent Maladaptive Behaviors\\
\years{2016}NIDA Director’s Travel Award, The College on Problems of Drug Dependence\\
\years{2015}Ruth L. Kirschstein Postdoctoral National Research Service Award, NIMH\\
\years{2015}Merit Abstract Award, Organization for Human Brain Mapping\\
\years{2015}Travel Award, Scientific Research Network on Decision Neuroscience \& Aging\\
\years{2009}Ruth L. Kirschstein Predoctoral National Research Service Award, NIMH\\
\years{2010}Travel Award, Organization for Human Brain Mapping\\
\years{2009}Fellow, Summer Institute in Cognitive Neuroscience, UC - Santa Barbara\\
\years{2008}Fellow, CIT Flexible Learning Space, Duke University\\
\years{2007}Travel Award, Organization for Human Brain Mapping\\
\years{2006}Roger Black Award for Psychological Research, University of South Carolina\\
\years{2005}\textit{Phi Beta Kappa}, University of South Carolina\\
\years{2005}Fellow, NSF Summer Research Institute, University of South Carolina\\
\years{2005}Stubbs Scholar, University of South Carolina\\
\years{2004}Phi Beta Kappa Freshman Scholar Award, University of South Carolina\\
\years{2004}Baroody Scholar Award, University of South Carolina\\
\years{2003}Abney Scholar, University of South Carolina\\
\years{2002}LIFE Scholarship, University of South Carolina\\


%\hrule \section*{Extramural Funding}
\section*{Grants}
[*approximate total costs; †approximate direct costs]
\subsection*{Active}
\years{2018-2020} \textbf{Social Reward Learning in Schizophrenia}. (NIH R21-MH116422; †\$275,000). Consultant, with PI Butler. Effort: n/a \\ [.2cm]
\years{2018-2019} \textbf{Modulating Individual Differences in Reward Sensitivity with Transcranial Current Stimulation}. (Targeted Small Grant Award from the Office of the Vice President for Research at Temple University; †\$10,000). Principal Investigator, with Co-I Reilly. Effort (unpaid): 11\% (Y1), 11\% (Y2) \\ [.2cm]
\years{2017-2020} \textbf{Remote Modulation of Reward Circuits with Noninvasive Brain Stimulation}. (NIH R21-MH113917; *\$420,000). Principal Investigator, with Co-I Krekelberg. Effort: 11\% (Y1), 25\% (Y2), 20\% (Y3) \\ [.2cm]
\years{2019-2021} \textbf{Pupillometry as a Physiological Biomarker for Preclinical Dementia in Minority Aging}. (Alzheimer’s Disease Administrative Supplement for NIH R01-DC013063; *\$388,000). Co-Investigator, with PI Reilly. Effort: 11\% (Y1) \\ [.2cm]
\years{2019-2020} \textbf{Aberrant Reward Sensitivity: Mechanisms Underlying Substance Use}. (NIH R03-DA046733; *\$225,000). Principal Investigator, with Co-Is Alloy, Jarcho, McCloskey, and Chein. Effort: 16.67\% (Y1)


\subsection*{Pending}
\years{2019-2024} \textbf{Parsing Reward with Corticostriatal Network Maps}. (NIH R01-DA048857; *\$2,000,000). Principal Investigator, with Co-Is Alloy, Delgado, and Olino. \\ [.2cm]
\years{2020-2025} \textbf{Neural Mechanisms of Social and Nonsocial Reward Processing in Adulthood: Identifying Risk Factors for Social Victimization and Alzheimer's Disease}. (NIH R01-AG067011; *\$3,750,000). Principal Investigator, with Co-Is Fareri, Giovannetti, Jarcho, Olson, Reilly, and Venkatraman.

\subsection*{Completed}
\years{2017-2018} \textbf{Social Reward and Aging: Identifying the Neural Underpinnings of Peer Influences}. (Pilot Grant from the Scientific Research Network on Decision Neuroscience \& Aging; †\$30,000). Principal Investigator, with Co-Is Fareri, Giovannetti, and Reeck. [Subaward of NIH R24-AG054355 (PI Samanez-Larkin)] Effort (unpaid): 11\% (Y1), 25\% (Y2) \\ [.2cm]
\years{2015-2016} \textbf{Parsing Reward: Identifying Distinct Neural Pathways for Specific Reward Properties}. (NIH F32-MH107175; †\$110,000). Principal Investigator. \\ [.2cm]
\years{2009-2012} \textbf{Neurobiological Underpinnings of Decision Making}. (NIH F31-MH086248; †\$82,000). Principal Investigator. \\


% PUBLICATIONS
\section*{Bibliography}
\aiGoogleScholar \hspace{.01cm} \href{https://scholar.google.com/citations?user=czV7OcQAAAAJ&hl=en}{scholar.google.com}: h-index: 19; i10-index: 25; total citations: 1656 \\ 
\lbrack*shared first authorship; †trainee under my supervision; RCR: \textsc{Relative Citation Ratio}\footnote{The \textsc{Relative Citation Ratio} (RCR) is a field- and time-normalized citation metric \href{http://journals.plos.org/plosbiology/article?id=10.1371/journal.pbio.1002541}{(Hutchins et al., 2016, \textit{PLoS Biology})}. NIH-funded papers are the benchmark for RCR: Any paper with RCR = 1.0 has an RCR higher than 50\% of NIH-funded papers. Recent papers and papers that are not indexed on PubMed will not have an RCR. All RCR values were extracted from the \href{https://icite.od.nih.gov/stats}{\textit{iCite} database} using \href{https://github.com/gpoore/pythontex}{\pytex} on \today.}]

\subsection*{Publications}

% Function for extracting RCR
\begin{pycode}
import requests
def get_rcr(pmid):
  response = requests.get("/".join(["https://icite.od.nih.gov/api","pubs",str(pmid)]))
  pub = response.json()
  if pub['relative_citation_ratio'] is not None:
    return round(pub['relative_citation_ratio'],2)
  else:
    return pub['relative_citation_ratio']
\end{pycode}


\begin{hangparas}{.5in}{1}
[37] Wang Y*, Metoki A*, \textbf{Smith DV}, Medaglia JD, Zang Y, Benear S, Popal H, Lin Y, Olson IR (2019, in press). Multimodal Mapping of the Face Connectome. \textit{Nature Human Behaviour}. RCR = None.

[36] Ng TH†, Alloy LB, \textbf{Smith DV} (2019, in press). Meta-analysis of Reward Processing in Major Depressive Disorder Reveals Distinct Abnormalities within the Reward Circuit. \textit{Translational Psychiatry}. RCR = None. Postprint available on \biorxiv{https://www.biorxiv.org/content/10.1101/332981v3}.

[35] Diehl MM, Lempert K, Parr AC, Ballard I, Steele VR, \textbf{Smith DV} (2018). Toward an Integrative Perspective on the Neural Mechanisms Underlying Persistent Maladaptive Behaviors. \textit{European Journal of Neuroscience}, 48(3), 1870-1883. RCR = \py{get_rcr(30044022)} \pdf{https://onlinelibrary.wiley.com/doi/epdf/10.1111/ejn.14083}

[34] Utevsky AV, \textbf{Smith DV}, Young JS, Huettel SA (2017). Large-Scale Network Coupling with the Fusiform Cortex Future Social Motivation. \textit{eNeuro}, 4(5): eneuro.0084-17.2017. RCR = \py{get_rcr(29034316)} \pdf{http://www.eneuro.org/content/eneuro/4/5/eneuro.0084-17.2017.full.pdf}

[33] Li R*, \textbf{Smith DV}*, Clithero JA, Venkatraman V, Carter RM, Huettel SA (2017). Reason’s Enemy is Not Emotion: Engagement of Cognitive Control Networks Explain Biases in Gain/Loss Framing. \textit{Journal of Neuroscience}, 37 (13) 3588-3598. RCR = \py{get_rcr(28264981)} \pdf{http://www.jneurosci.org/content/jneuro/37/13/3588.full.pdf}

[32] \textbf{Smith DV} \& Delgado MR (2017). Meta-Analysis of Psychophysiological Interactions: Revisiting Cluster-Level Thresholding and Sample Sizes. \textit{Human Brain Mapping}, 38(1), 588-591. RCR = \py{get_rcr(27543687)} \html{https://www.ncbi.nlm.nih.gov/pubmed/27543687}

[31] Cho C, \textbf{Smith DV}, Delgado MR (2016). Reward Sensitivity Enhances Ventrolateral Prefrontal Cortex Activation During Free Choice. \textit{Frontiers in Neuroscience}, 10:529. RCR = \py{get_rcr(27917106)} \html{https://www.ncbi.nlm.nih.gov/pmc/articles/PMC5114280/}

[30] \textbf{Smith DV}, Gseir M, Speer ME, Delgado MR (2016). Toward a Cumulative Science of Functional Integration: a Meta-Analysis of Psychophysiological Interactions. \textit{Human Brain Mapping}, 37(8), 2904-17. RCR = \py{get_rcr(27145472)} \pdf{http://nwkpsych.rutgers.edu/neuroscience/publications/2016_SmithDelgado_HBM.pdf}

[29] \textbf{Smith DV}, Rigney AE, Delgado MR (2016). Distinct Reward Properties are Encoded via Corticostriatal Interactions. \textit{Scientific Reports}, 6, 20093. RCR = \py{get_rcr(26831208)} \pdf{https://www.nature.com/articles/srep20093.pdf}

[28] Bhanji JP, \textbf{Smith DV}, Delgado MR (2016). A Brief Anatomical Sketch of Human Ventromedial Prefrontal Cortex. [Supplementary Note 1 for Delgado et al. (2016). \textit{Nature Neuroscience}, 19(12), 1545-1552]. RCR = None \pdf{https://media.nature.com/original/nature-assets/neuro/journal/v19/n12/extref/nn.4438-s1.pdf}. Postprint available on \psyarxiv{https://psyarxiv.com/zdt7f/}.

[27] Wang KS, \textbf{Smith DV}, Delgado MR (2016). Using fMRI to Study Reward Processing in Humans: Past, Present, and Future. \textit{Journal of Neurophysiology}, 115, 1664-1678. RCR = \py{get_rcr(26740530)} \pdf{http://www.physiology.org/doi/pdf/10.1152/jn.00333.2015}

[26] \textbf{Smith DV} \& Delgado MR (2015). Reward Processing. In A. W. Toga (Ed.), \textit{Brain Mapping: An Encyclopedic Reference} (1st ed., pp. 361-366). Waltham, MA: Academic Press. RCR = None \pdf{http://nwkpsych.rutgers.edu/neuroscience/publications/2015_SmithDelgado_RewardChapter.pdf}. Postprint available on \psyarxiv{https://psyarxiv.com/b3gea/}.

[25] \textbf{Smith DV}*, Sip KE*, Delgado MR (2015). Functional Connectivity with Distinct Neural Networks Tracks Fluctuations in Gain/Loss Framing Susceptibility. \textit{Human Brain Mapping}, 36(7), 2743-55. RCR = \py{get_rcr(25858445)} \pdf{http://nwkpsych.rutgers.edu/neuroscience/publications/2015_SmithSipDelgado_HBM.pdf}

[24] Young JS*, \textbf{Smith DV}*, Coutlee CG, Huettel SA (2015). Synchrony Between Sensory and Cognitive Networks is Associated with Subclinical Variation in Autistic Traits. \textit{Frontiers in Human Neuroscience}, 9:146. RCR = \py{get_rcr(25852527)} \html{https://www.frontiersin.org/articles/10.3389/fnhum.2015.00146/full}

[23] Sip KE, \textbf{Smith DV}, Porcelli AJ, Kar K, Delgado MR (2015). Social Closeness and Feedback Modulate Susceptibility to the Framing Effect. \textit{Social Neuroscience}, 10(1), 35-45. RCR = \py{get_rcr(25074501)} \pdf{http://nwkpsych.rutgers.edu/neuroscience/publications/2014_SipSmithDelgado_SocNeuro.pdf}

[22] \textbf{Smith DV} \& Delgado MR (2015). Social Nudges: Utility Conferred from Others. \textit{Nature Neuroscience}, 18(6), 791-792. RCR = \py{get_rcr(26007210)} \pdf{http://nwkpsych.rutgers.edu/neuroscience/publications/2015_SmithDelgado_NatNeuro.pdf}

[21] Murty VP, Shermohammed M, \textbf{Smith DV}, Carter RM, Huettel SA, Adcock RA (2014). Resting State Networks Distinguish Human Ventral Tegmental Area from Substantia Nigra. \textit{NeuroImage}, 100(1), 580-589. RCR = \py{get_rcr(24979343)} \html{http://www.sciencedirect.com/science/article/pii/S1053811914005242}

[20] \textbf{Smith DV}, Utevsky AV, Bland AR, Clement NJ, Clithero JA, Harsch AE, Carter RM, Huettel SA (2014). Characterizing Individual Differences in Functional Connectivity Using Dual-Regression and Seed-Based Approaches. \textit{NeuroImage}, 95(1), 1-12. RCR = \py{get_rcr(24662574)} \html{http://www.sciencedirect.com/science/article/pii/S105381191400202X}

[19] \textbf{Smith DV}, Clithero JA, Boltuck SE, Huettel SA (2014). Functional Connectivity with Ventromedial Prefrontal Cortex Reflects Subjective Value for Social Rewards. \textit{Social Cognitive and Affective Neuroscience}, 9(12), 2017-2025. RCR = \py{get_rcr(24493836)} \html{https://www.ncbi.nlm.nih.gov/pubmed/24493836}

[18] Utevsky AV, \textbf{Smith DV}, Huettel SA (2014). Precuneus is a Functional Core of the Default-Mode Network. \textit{Journal of Neuroscience}, 34(3), 932-940. RCR = \py{get_rcr(24431451)} \pdf{http://www.jneurosci.org/content/jneuro/34/3/932.full.pdf}

[17] Karnath H-O \& \textbf{Smith DV} (2014). The Next Step in Modern Brain Lesion Analysis: Multivariate Pattern Analysis. \textit{Brain}, 137(9), 2405-2407. RCR = \py{get_rcr(25125587)} \pdf{https://sites.google.com/a/temple.edu/dvs-lab/2014_KarnathSmith_Brain.pdf}

[16] Strauman TJ, Detloff AM, Sestokas R, \textbf{Smith DV}, Goetz EL, Rivera C, Kwapil L (2013). What Shall I Be, What Must I Be: Neural Correlates of Personal Goal Activation. \textit{Frontiers in Integrative Neuroscience}, 6:123. RCR = \py{get_rcr(23316145)} \html{https://www.ncbi.nlm.nih.gov/pmc/articles/PMC3539852/}

[15] \textbf{Smith DV}, Clithero JA, Rorden C, Karnath H-O (2013). Decoding the Anatomical Network of Spatial Attention. \textit{Proceedings of the National Academy of Sciences of the USA}, 110(4), 1518-1523. RCR = \py{get_rcr(23300283)} \pdf{http://www.pnas.org/content/110/4/1518.full.pdf}

[14] Jelsone-Swain L, \textbf{Smith DV}, Baylis GC (2012). The Effect of Stimulus Duration and Motor Response in Hemispatial Neglect During a Visual Search Task. \textit{PLoS ONE}, 7(5), e37369. RCR = \py{get_rcr(22662149)} \html{http://journals.plos.org/plosone/article?id=10.1371/journal.pone.0037369}

[13] Libedinsky C, \textbf{Smith DV}, Teng CS, Namburi P, Chen V, Huettel SA, Chee MLW (2011). Sleep Deprivation Alters Valuation Signals in the Ventromedial Prefrontal Cortex. \textit{Frontiers in Behavioral Neuroscience}, 5:70. RCR = \py{get_rcr(22028686)} \html{https://www.ncbi.nlm.nih.gov/pmc/articles/PMC3199544/}

[12] Clithero JA, Reeck CC, Carter RM, \textbf{Smith DV}, Huettel SA (2011). Nucleus Accumbens Mediates Relative Motivation for Rewards in the Absence of Choice. \textit{Frontiers in Human Neuroscience}, 5:87. RCR = \py{get_rcr(21941472)} \html{https://www.ncbi.nlm.nih.gov/pmc/articles/PMC3171065/}

[11] Bland AR, Mushtaq F, \textbf{Smith DV} (2011). Exploiting Trial-to-Trial Variability in Multimodal Experiments. \textit{Frontiers in Human Neuroscience}, 5:80. RCR = \py{get_rcr(21886619)} \html{https://www.ncbi.nlm.nih.gov/pmc/articles/PMC3155870/}

[10] Appelbaum LG, \textbf{Smith DV}, Boehler CN, Wen C, Woldorff MG (2011). Rapid Modulation of Sensory Processing Induced by Stimulus Conflict. \textit{Journal of Cognitive Neuroscience}, 23(9), 2620-2628. RCR = \py{get_rcr(20849233)} \html{https://www.ncbi.nlm.nih.gov/pmc/articles/PMC3096678/}

[9] Clithero JA, \textbf{Smith DV}, Carter RM, Huettel SA (2011). Within- and Cross-Participant Classifiers Reveal Different Neural Coding of Information. \textit{NeuroImage}, 56(2), 699-708. RCR = \py{get_rcr(20347995)} \html{https://www.ncbi.nlm.nih.gov/pmc/articles/PMC2908207/}

[8] \textbf{Smith DV} \& Huettel SA (2010). Decision Neuroscience: Neuroeconomics. \textit{Wiley Interdisciplinary Reviews: Cognitive Science}, 1(6), 854-871. RCR = \py{get_rcr(22754602)} \html{https://www.ncbi.nlm.nih.gov/pmc/articles/PMC3384699/}

[7] Hayden BY, \textbf{Smith DV}, Platt ML (2010). Cognitive Control Signals in Posterior Cingulate Cortex. \textit{Frontiers in Human Neuroscience}, 4:223. RCR = \py{get_rcr(21160560)} \html{https://www.ncbi.nlm.nih.gov/pmc/articles/PMC3001991/}

[6] \textbf{Smith DV}, Davis B, Niu K, Healy E, Bonilha L, Fridriksson J, Morgan P, Rorden C (2010). Spatial Attention Evokes Similar Activation Patterns for Visual and Auditory Stimuli. \textit{Journal of Cognitive Neuroscience}, 22(2), 347-361. RCR = \py{get_rcr(19400684)} \html{https://www.ncbi.nlm.nih.gov/pmc/articles/PMC2846529/}

[5] \textbf{Smith DV}, Hayden BY, Truong T-K, Song AW, Platt ML, Huettel SA (2010). Distinct Value Signals in Anterior and Posterior Ventromedial Prefrontal Cortex. \textit{Journal of Neuroscience}, 30(7), 2490-2495. RCR = \py{get_rcr(20164333)} \html{https://www.ncbi.nlm.nih.gov/pmc/articles/PMC2856318/}

[4] Hayden BY, \textbf{Smith DV}, Platt ML (2009). Electrophysiological Correlates of Default-Mode Processing in Macaque Posterior Cingulate Cortex. \textit{Proceedings of the National Academy of Sciences of the USA}, 106(14), 5948-5953. RCR = \py{get_rcr(19293382)} \html{Electrophysiological Correlates of Default-Mode Processing in Macaque Posterior Cingulate Cortex}

[3] \textbf{Smith DV} \& Clithero JA (2009). Manipulating Executive Function with Transcranial Direct Current Stimulation. \textit{Frontiers in Integrative Neuroscience}, 3:26. RCR = \py{get_rcr(19847324)} \html{https://www.ncbi.nlm.nih.gov/pmc/articles/PMC2764379/}

[2] Clithero JA \& \textbf{Smith DV} (2009). Reference and Preference: How Does the Brain Scale Subjective Value? \textit{Frontiers in Human Neuroscience}, 3:11. RCR = \py{get_rcr(19680434)} \html{https://www.ncbi.nlm.nih.gov/pmc/articles/PMC2715285/}

[1] Almor A, \textbf{Smith DV}, Bonilha L, Fridriksson J, Rorden C (2007). What is in a Name? Spatial Brain Circuits are Used to Track Discourse References. \textit{Neuroreport}, 18(12), 1215-1219. RCR = \py{get_rcr(17632270)} \pdf{http://www.mccauslandcenter.sc.edu/aLab/sites/sc.edu.alab/files/attachments/almor-et-al-2007.pdf} \\

\end{hangparas}


\subsection*{Preprints Under Review}

\begin{hangparas}{.5in}{1}

Wang S†*, Taren AA*, \textbf{Smith DV}. Functional Parcellation of the Default Mode Network: A Large-Scale Meta-Analysis. Preprint available on \biorxiv{https://www.biorxiv.org/content/early/2017/11/27/225375}.

Chen EY, Murray SM, Giovannetti T, \textbf{Smith DV} (under review). Reduced gray matter volume in the orbitofrontal cortex is associated with greater body mass index: a coordinate-based meta-analysis. Preprint available on \biorxiv{https://www.biorxiv.org/content/early/2018/06/30/359919}. \\

\end{hangparas}


\subsection*{Manuscripts in Preparation or Under Review}

\begin{hangparas}{.5in}{1}


\textbf{Smith DV}, Kragel PA, Clithero JA, Revill KP, Rorden C, Huettel SA, Carter RM (in prep). Medial-Lateral Gradient within the Human Striatum Decodes Social Rewards.

%Lewis AH, \textbf{Smith DV}, Manglani H, Delgado MR (in prep). Neural Activation and Functional Connectivity During Extinction Learning with Appetitive and Aversive Conditioned Stimuli.

Kim ES, Wang KS, \textbf{Smith DV}, Speer ME, Delgado MR (in prep). Neural Correlates of Self-Evaluation Enhancement and Dishonest Decisions.

\textbf{Smith DV}, Wang KS, Delgado MR (in prep). Distinct Spatiotemporal Patterns within the Human Striatum Distinguish Reward and Punishment.

Dobryakova E \& \textbf{Smith DV} (in prep). Reward Enhances Connectivity between the Ventral Striatum and the Default Mode Network.

Tepfer LJ†, Alloy LB,  \textbf{Smith DV} (in prep). Familial and lifetime history of depression: alterations in the neural circuitry underlying reward and social cognition.

Fareri DS, \textbf{Smith DV}, Delgado MR (in prep). The role of relationship closeness on network connectivity during trust-based interactions.

\end{hangparas}


%\hrule \section*{Invited Talks}
\vspace{.4cm}
\section*{Invited Talks}
\begin{hangparas}{.5in}{1}

\years{2018} Constructing Value: Understanding the Role of Corticostriatal Connectivity. Symposium on Biology of Decision-Making in Paris, France.

\years{2017} Social and Economic Rewards Enhance Connectivity between the Ventral Striatum and the Default Mode Network. Rutgers University---Camden.

\years{2016} Brain Connectivity Shapes Responses to Social and Economic Incentives. The Nathan S. Kline Institute for Psychiatric Research, New York University.

\years{2016} Neural Circuitry Underlying Social and Economic Incentives. Temple University.

\years{2015} Neural Circuitry Underlying Social and Economic Incentives. Bard College.

\years{2015} Characterizing Individual Differences in Brain Connectivity. Kessler Foundation.

\years{2015} Linking Neural Circuits to Social and Economic Incentives: From Valuation to Outcome. Dartmouth College.

\years{2015} Interacting Brain Regions Contribute to a Range of Individual Differences. Kessler Foundation.

\years{2014} Advanced Statistical Procedures in Lesion Analysis: Multivariate Pattern Analysis. Federation of the European Societies of Neuropsychology Summer School in Berlin, Germany.

\years{2014} Characterizing Individual Differences in Decision Making. Sackler Institute for Developmental Psychobiology, Weill Medical College of Cornell University.

\years{2011} Neural Mechanisms of Social Valuation. Rutgers University---Newark.

\years{2011} Neural Mechanisms of Social Valuation. University of South Carolina.

\years{2010} Using FSL for Basic and Advanced Neuroimaging Analyses. Georgia State University / Georgia Tech Center for Advanced Brain Imaging. \\

\end{hangparas}



 %\SERVICE}
\vspace{.2cm}
\section*{Service and Professional Activities}

\subsection*{Journal Reviewing \& Editorial Roles}
\years{2018-present}Academic Editor, \textit{PLoS ONE}. \\
\years{2015-2017}Review Editor, \textit{Frontiers in Psychology}, section Decision Neuroscience. \\
\years{2015-2017}Review Editor, \textit{Frontiers in Neuroscience}, section Decision Neuroscience. \\

Ad Hoc Reviewer:
\begin{multicols}{2}
\begin{itemize}[noitemsep]
\itshape
\item Advances in Methods and Practices in Psychological Science
\item Annals of the New York Academy of Sciences
\item Brain
\item Brain Structure and Function
\item BMC Neuroscience 
\item Cell Reports
\item Cerebral Cortex 
\item Cognitive, Affective, and Behavioral Neuroscience 
\item Cortex
\item Current Directions in Psychological Science
\item Developmental Neuroscience 
\item European Journal of Neurology
\item Frontiers in Human Neuroscience 
\item Frontiers in Neuroinformatics 
\item Frontiers in Neurology 
\item Frontiers in Neuroscience 
\item Frontiers in Psychology 
\item Human Brain Mapping 
\item International Journal of Environmental Research and Public Health
\item International Journal of Hyperthermia 
\item Journal of Cognitive Neuroscience 
\item Journal of Neuroscience 
\item Journal of Neuroscience, Psychology, \& Economics
\item Management Information Systems Quarterly 
\item Nature Communications 
\item Nature Human Behaviour 
\item Nature Neuroscience
\item NeuroImage 
\item Neuroimage: Clinical 
\item Neuropsychologia 
\item Neuroscience Research 
\item PLoS Biology 
\item PLoS ONE 
\item Proceedings of the National Academy of Sciences of the USA
\item Psychological Science 
\item Psychopathology 
\item Social, Cognitive, and Affective Neuroscience 
\item Social Neuroscience
\end{itemize}
\end{multicols}

\aiPublons \hspace{.05cm} \href{https://publons.com/author/1204254/david-v-smith}{publons.com} contains complete reviewing and editing record. \\ [.2cm]
Awards on \textit{Publons}: \\
\years{2018} \href{https://publons.com/awards/2018/esi/?name=David\%20V.\%20Smith&esi=15}{Top Reviewers for Neuroscience \& Behavior} \\
\years{2018} \href{https://publons.com/awards/2018/esi/?name=David\%20V.\%20Smith&esi=22}{Top Reviewers for Multidisciplinary} \\
\years{2017} \href{https://publons.com/awards/institution/?asjc=72&order_by=place}{Top Reviewers for Temple University} \\
\years{2017} \href{https://publons.com/awards/field/?name=David\%20V.\%20Smith\&asjc=100}{Top Reviewers for Neuroscience}


\subsection*{Grant Reviewing}
\years{2019} National Science Foundation, Social Psychology Program. \\
\years{2019} Mind Science Foundation. \\
\years{2018} National Science Foundation, Social Psychology Program. \\
\years{2018} Scientific Research Network on Decision Neuroscience \& Aging, Pilot Grants. \\
\years{2017} FWF Austrian Science Fund, START Program. \\
\years{2017} Swiss National Science Foundation, Humanities and Social Sciences, Division I. \\
\years{2017} Wellcome Trust, Senior Research Fellowship in Basic Biomedical Science. \\
\years{2015} Israel Science Foundation, Individual Research Grant. \\
\years{2014} Scientific Research Network on Decision Neuroscience \& Aging, Pilot Grants.

\subsection*{Departmental Service at Temple University}
\years{2018-present} Co-organizer (with Dr. Mathieu Wimmer), Maladaptive Motivated Behaviors Seminar. \\
\years{2018-2019} Member, Undergraduate Neuroscience Committee. \\ [.1cm]
\years{2017-present} Member, Statistics Curriculum Committee. \\ 
\years{2017-present} Organizer, Neuroimaging Methods Journal Club. \\
\years{2017-2018} Member, Faculty Search Committee in Social / Affective Neuroscience. \\
\years{2017} Member, Subcommittee to evaluate and design deep learning server. \\
\years{2016-2017} Member, Faculty Search Committee in Cognitive / Cognitive Neuroscience.


\subsection*{Society Memberships}
\begin{multicols}{2}
\begin{itemize}[noitemsep]
\item Association for Psychological Science
\item Cognitive Neuroscience Society
\item Eastern Psychological Association
\item New York Academy of Sciences
\item Organization for Human Brain Mapping
\item Social \& Affective Neuroscience Society
\item Society for Neuroeconomics
\item Society for Neuroscience
\item Society for Social Neuroscience
\item Society of Biological Psychiatry
\end{itemize}
\end{multicols}

\subsection*{Other Activities}
\begin{hangparas}{.5in}{1}
Co-Organizer, Duke University Neuroeconomics Journal Club (2008-2009).

Conference Abstract Reviewer: Organization for Human Brain Mapping (2010, 2013-2019); Cognitive Science Society (2017).

\end{hangparas}


% \section*{Contributions to Open Science}
\vspace{.4cm}
\section*{Contributions to Open Science}
\subsection*{Statistical Maps}
\begin{itemize}[noitemsep]
\item \href{http://neurovault.org/collections/1484/}{Li*, Smith*, et al. (2017), \textit{Journal of Neuroscience}}
\item \href{http://neurovault.org/collections/1408/}{Smith et al. (2016), \textit{Scientific Reports}}
\item \href{http://neurovault.org/collections/1406/}{Smith et al. (2016), \textit{Human Brain Mapping}}
\item \href{http://neurovault.org/collections/2132/}{Cho et al. (2016), \textit{Frontiers in Neuroscience}}
\item \href{http://neurovault.org/collections/2485/}{Murty et al. (2014), \textit{NeuroImage}}

\end{itemize}

\subsection*{Analysis Code}
\begin{itemize}[noitemsep]
\item \faGithub \hspace{.2cm} \href{https://github.com/edobryakova/DobryakovaSmith_HCP}{Dobryakova \& Smith (in prep)}
\item \faGithub \hspace{.2cm} \href{https://github.com/DVS-Lab/dmn-parcellation}{Wang*, Taren*, \& Smith (under review)} \\
\end{itemize}


% TEACHING
\section*{Teaching Activities}
% \subsection*{Instructor of Record at Temple University}
% [\underline{U}ndergraduate, \underline{G}raduate; \underline{S}pring, \underline{F}all; *original course] \\ [.2cm]
% Topics: Brain, Behavior and Cognition (``Decision Making and the Brain") [U]: 2017F
% [*\tiny{shared first authorship}; †\tiny{trainee in Smith Lab}
\subsection*{Teaching Experience}

[\underline{U}ndergraduate, \underline{G}raduate; \underline{S}pring, \underline{F}all; *original course] \\ [.2cm]
Instructor, \textit{Current Topics in Neuroscience} [U], Temple University: 2019F\footnote{Cross-listed with \textit{Topics in Psychology}; Topic: ``Neuroimaging: From Image to Inference"}

Instructor, \textit{Decision Neuroscience}* [U], Temple University: 2019S

Instructor, \textit{Foundations of Sensation and Perception} [U], Temple University: 2018F

Guest Lecturer, \textit{Advanced Neuroanatomy} [G], Lewis Katz School of Medicine, Temple University: 2018S

Instructor, \textit{Topics: Brain, Behavior and Cognition} [U], Temple University: 2017F\footnote{Topic: ``Decision Making and the Brain"}

Guest Lecturer, \textit{The Emotional Brain} [U], Rutgers University: 2014S

Lab Instructor, \textit{Neuroscience Boot Camp} [G], Duke University: 2011F

Teaching Assistant, \textit{Introduction to Cognitive Neuroscience} [U], Duke University: 2010S

Teaching Assistant, \textit{Functional Magnetic Resonance Imaging} [G], Duke University: 2008F

Teaching Assistant, \textit{Brain Waves and Cognition} [U], Duke University: 2008S

Supplemental Instruction Leader, \textit{Psychological Statistics} [U], University of South Carolina: 2006S

Teaching Assistant, \textit{Introductory Psychology} [U], University of South Carolina: 2005F, 2006S \\



\subsection*{Mentoring}

\begin{tabbing}
Graduate students, as primary or co-primary mentor: \\
\hspace{.5in} \= Jeffrey Dennison (2018-present), Ph.D. in Psychology (C \& N), expected 2023 \\
\> Karen Shen (2019-present), Ph.D. in Psychology (C \& N), expected 2024 (co-mentored with Murty) \\
\> Daniel Sazhin (2019-present), Ph.D. in Psychology (C \& N), expected 2024 \\ [.2cm]

Graduate students (other), as committee member or secondary mentor: \\
\> Tommy Ng (2017-present, Clinical Psychology) \\
\> Iris Chat (2018-present, Clinical Psychology) \\
\> Corinne Bart (2019-present, Clinical Psychology) \\
\> Katherine Hackett (2018-present, Clinical Psychology) \\
\> Nicole Henninger (2018-present, Communications) \\
\> Dr. Sangsuk Yoon (2018, Business School) \\
\> Dr. William Hampton (2017-2018) \\
\> Dr. Ashley Drew (2017) \\
\> Dr. Kylie Alm (2017) \\
\> Dr. Gail Rosenbaum (2017) \\
\> Dr. Trishala Parthasarathi (2016-2017, University of Pennsylvania) \\ [.2cm]



%Undergraduates and post-baccalaureate research assistants: \\
Advisor for undergraduate independent study projects, visiting scholars, internships, research \\ 
assistantships, and research awards. Also advisor for post-baccalaureate research assistants. \\
Selected examples below: \\

\> Adam Lang, Neuroscience. \\
\> John Marc Cipriaso, Honors Psychology. \\
\> Dennis DeSalme, Psychology. \\
\> Caleb Haynes (research assistantship, '19-present), co-mentored with Jarcho. \\
\> Srikar Katta, Honors Economics ('19 \textit{LAURA Scholar}). \\
\> Victoria Kelly (research assistantship, '18-present). \\
\> Rachael Kinmartin, Psychology. \\
\> Lindsey Tepfer (research assistantship, '19-present; Temple '19 Masters in Neuroscience). \\
\> Brijai Varma (visiting scholar, summer '18, '19), University of Pittsburgh. \\
\> Isaac Levy (visiting scholar, summer '18), Oberlin College. \\
\> Benjamin Muzekari (Temple '19; '18 \textit{LAURA Scholar}, '19 \textit{LAURA Scholar}), Honors Psychology. \\
\> Jane Gaisinsky (Temple '19; '17 \textit{LAURA Scholar}, '18 \textit{LAURA Scholar}), Neuroscience. \\
\> Michael Fitzpatrick (Temple '18), Economics. \\
\> Shaoming Wang (research assistantship, '17-18). \\
\> Christian Reice (Temple, '17), Neuroscience.



\end{tabbing}

\pagebreak
\begin{hangparas}{.5in}{1}
Prior trainees\footnote{These were individuals who I mentored in neuroimaging analysis while under the supervision of S. Huettel or M. Delgado.} with coauthored publications: Jacob S. Young (Duke undergrad); Sarah Boltuck (Duke undergrad); Amanda Utevsky (Duke grad); Rosa Li (Duke grad); Amy Bland (visiting Duke grad, from U of Manchester); Catherine Cho (Rutgers grad); K. Sally Wang (Rutgers grad); Mouad Gseir (Rutgers undergrad). \\ [.1cm]
\end{hangparas}



% RECENT CONFERENCE PRESENTATIONS
\section*{Recent Conference Presentations}

\subsection*{2019}
\begin{hangparas}{.5in}{1}

Tepfer LJ, Alloy LB, \textbf{Smith DV} (October, 2019). Familial and lifetime history of depression: alterations in the neural circuitry underlying reward and social cognition. Poster to be presented at the 50th meeting of the Society for Neuroscience. Chicago, IL, USA.

Hackett K, Henninger NM, Kelly V, Giovannetti T, Fareri DS, \textbf{Smith DV} (October, 2019). Response to perceived fairness is associated with reduced connectivity within reward circuitry in older adults. Poster to be presented at the 50th meeting of the Society for Neuroscience. Chicago, IL, USA.

Kelly V, Hackett K, Henninger NM, Giovannetti T, \textbf{Smith DV}, Fareri DS (October, 2019). Aging alters corticostriatal interactions during shared reward processing. Poster to be presented at the 50th meeting of the Society for Neuroscience. Chicago, IL, USA.

\textbf{Smith DV}, Liu Y, Krekelberg B (October, 2019). Transcranial alternating current stimulation alters reward-dependent corticostriatal interactions. Poster to be presented at the 50th meeting of the Society for Neuroscience. Chicago, IL, USA.

Henninger NM, Kelly V, Hackett K, Fareri DS, Tepfer LJ, Katta S, Reeck C, Giovannetti T, Beard EC, Dennison J, Muzekari B, Desalme DF, Kinmartin R, Lang A, Cipriaso JM, Hunter E, Morrison C, \textbf{Smith DV} (October, 2019). Age‐related reductions in functional connectivity in social brain systems during an economic trust task. Poster to be presented at the 50th meeting of the Society for Neuroscience. Chicago, IL, USA.

Dennison JB, Ng T, Alloy L, \textbf{Smith DV} (October, 2019). Using Corticostriatal Networks to Disentangle Reward Value and Salience. Poster to be presented at the 50th meeting of the Society for Neuroscience. Chicago, IL, USA.

Tepfer LJ, Slipenchuk M, Muzekari B, Krekelberg B, \textbf{Smith DV} (June, 2019). Altering Social Norm Compliance with Transcranial Alternating Current Stimulation. Poster presented at the 9th meeting of the Interdisciplinary Symposium on Decision Neuroscience. Durham, North Carolina, USA.

Kelly V, Slipenchuk M, Katta S, Clithero JA, \textbf{Smith DV} (June, 2019). The More the Merrier: Participants Value Having More Options to Choose From. Poster presented at the 9th meeting of the Interdisciplinary Symposium on Decision Neuroscience. Durham, North Carolina, USA.

Henninger NM, Katta S, Kelly V, Hackett K,  Reeck C, Giovannetti T, Fareri DS, \textbf{Smith DV} (June, 2019). Aging is associated with reductions in functional connectivity in social brain systems. Abstract presented at the annual Interdisciplinary Symposium on Decision Neuroscience. Durham, NC, USA.

Fareri DS, Kelly V, Henninger NM, Hackett K, DeSalme D, Muzekari B, Katta S, Reeck C, Giovannetti T, \textbf{Smith DV} (May 2019). The influence of close relationships on shared reward processing in older and younger adults. Poster presented at the 12th meeting of the Social \& Affective Neuroscience Society. Miami, FL, USA.

\textbf{Smith DV}, Henninger NM, Hackett K, Kelly V, DeSalme D, Muzekari B, Katta S, Giovannetti T, Fareri DS (June, 2019). Fairness is Associated with Increased Connectivity between the Executive Control Network and MPFC. Abstract submitted for consideration at the 25th meeting of the Organization for Human Brain Mapping. Rome, Italy.

Ng TH, Alloy LB, \textbf{Smith DV} (March, 2019). Reward Processing in Preadolescents with Bipolar Disorder: An fMRI Study. Abstract submitted for consideration at the 21st meeting of the International Society for Bipolar Disorders. Sydney, Australia.

Tepfer LJ, Slipenchuk M, Muzekari B, Krekelberg B, \textbf{Smith DV} (March, 2019). Altering Social Norm Compliance with Transcranial Alternating Current Stimulation. Poster presented at the 90th meeting of the Eastern Psychological Association. New York, New York, USA.

Muzekari B, Slipenchuk M, Tepfer LJ, Krekelberg B, \textbf{Smith DV} (March, 2019). Modulating Social Influences on Fairness Perception with Transcranial Alternating Current Stimulation. Poster presented at the 90th meeting of the Eastern Psychological Association. New York, New York, USA. \\

\end{hangparas}


\subsection*{2018}
\begin{hangparas}{.5in}{1}

Chiu M, Ng TH, Alloy LB, \textbf{Smith DV} (October, 2018). Linking Valuation Circuitry with Maladaptive Decision Making within the Human Connectome Project. Poster to be presented at the 16th meeting of the Society for Neuroeconomics. Philadelphia, PA, USA.

Chiu M, Ng TH, Alloy LB, \textbf{Smith DV} (May, 2018). Reward-dependent Connectivity with Orbitofrontal Cortex in Subclinical Depression. Poster presented at the 73rd meeting of the Society of Biological Psychiatry. New York, NY, USA.

Zhang H, Venkatraman V, \textbf{Smith DV} (May, 2018). Perceiving Social Interactions Suppresses Connectivity between the Default Mode Network and Ventral Striatum. Poster presented at the 11th meeting of the Social \& Affective Neuroscience Society. New York, NY, USA. \\

\end{hangparas}


\subsection*{2017}
\begin{hangparas}{.5in}{1}
Ng TH, Alloy LB, \textbf{Smith DV} (November, 2017). Reward Processing Abnormalities in Mood Disorders: A Systematic Review and Meta-analysis of Neuroimaging Studies. Poster presented at the 51st meeting of the Association for Behavioral and Cognitive Therapies. San Diego, CA, USA.

Ng TH, Alloy LB, \textbf{Smith DV} (October, 2017). Reward Processing Abnormalities in Unipolar Depression: A Meta-analysis of Neuroimaging Studies. Poster presented at the 15th meeting of the Society for Neuroeconomics. Toronto, ON, Canada.

Wang S, \textbf{Smith DV}, Delgado MR (September, 2017). Informative and Affective Neural Pathways underlying Explore-Exploit Tradeoffs. Poster presented at the inaugural Computational Cognitive Neuroscience conference. New York, NY, USA.

Dobryakova E \& \textbf{Smith DV} (June, 2017). Reward Enhances Connectivity between the Ventral Striatum and the Default Mode Network. Poster presented at the 23rd meeting of the Organization for Human Brain Mapping. Vancouver, BC, Canada.

Wang  S, Taren AA, \textbf{Smith DV} (May, 2017). Large-Scale Meta-Analytic Characterization of the Default Mode Network. Poster presented at the 29th meeting of the Association for Psychological Science. Boston, MA, USA.

Chen EY, Foster GD, Mohamed FB, Conklin CJ, Hoge WS, Olson IR, Chein JM, \textbf{Smith DV}, McCloskey MS, Obradović Z, Olino TM, and the Temple Eating Disorders program represented by Jean M. Arlt (May, 2017). Can Baseline Resting State Functional Connectivity Classify Clinically Significant Weight Loss 3 and 15 Months Later? Poster presented at the 29th meeting of the Association for Psychological Science. Boston, MA, USA.

Chen EY, Olson I, \textbf{Smith DV}, Olino T, McCloskey MS, Chein J, Edwards M, Obradovic Z  (April, 2017). The use of multivariate pattern analysis to develop and test an objective diagnostic clinical test for binge eating disorder in the context of obesity. Poster presented the 3rd meeting of the Association for Clinical and Translational Science. Washington, DC.

Fareri DS, \textbf{Smith DV}, Delgado MR (March, 2017). Reciprocation from a friend enhances coupling between the default mode network and ventral striatum. Talk given (by D. Fareri) at the 10th meeting of the Social \& Affective Neuroscience Society. Los Angeles, CA, USA. \\

\end{hangparas}


\subsection*{2016}
\begin{hangparas}{.5in}{1}
\textbf{Smith DV}, Wang S, Delgado MR (November, 2016). Neural Pathways Underlying Explore-Exploit Tradeoffs in Social and Nonsocial Contexts. Poster presented at the 7th meeting of the Society for Social Neuroscience. San Diego, CA, USA.

\textbf{Smith DV}, Wang S, Delgado MR (November, 2016). Neural Pathways Underlying Explore-Exploit Tradeoffs in Social and Nonsocial Contexts. Poster presented at the 46th meeting of the Society for Neuroscience. San Diego, CA, USA.

Li R, \textbf{Smith DV}, Clithero JA, Venkatraman V, Carter RM, Huettel SA (August, 2016). Revisiting the dual-systems model of choice using fMRI: Cognitive engagement and disengagement explain biases in gain/loss framing. Poster presented at the 14th meeting of the Society for Neuroeconomics. Berlin, Germany.

Utevsky AV, \textbf{Smith DV}, Venkatraman V, Huettel SA (August, 2016). Distinct subregions within the temporoparietal junction and posterior cingulate uniquely track prosocial decision-making. Poster presented at the 14th meeting of the Society for Neuroeconomics. Berlin, Germany.

Hakimi S, Clithero JA, Mullette-Gillman OA, \textbf{Smith DV}, McLaurin E, Taren A, Venkatraman V, Huettel SA, Carter RM (August, 2016). Decomposing Risk Representation in Parietal Cortex. Poster presented at the 14th meeting of the Society for Neuroeconomics. Berlin, Germany.

\textbf{Smith DV}, Li R, Clithero JA, Venkatraman V, Carter RM, Huettel SA (May, 2016). Revisiting the dual-systems model of choice using fMRI: Cognitive engagement and disengagement explain biases in gain/loss framing. Poster presented at the 28th meeting of the Association for Psychological Science. Chicago, IL, USA.

\textbf{Smith DV}, Gseir M, Speer ME, Delgado MR (April, 2016). Toward a Cumulative Science of Functional Integration: a Meta-Analysis of Psychophysiological Interactions. Poster presented at the 9th meeting of the Social \& Affective Neuroscience Society. New York, NY, USA.

\textbf{Smith DV}, Gseir M, Speer ME, Delgado MR (April, 2016). Toward a Cumulative Science of Functional Integration: a Meta-Analysis of Psychophysiological Interactions. Poster presented at the 10th annual Reprogramming the Brain to Health symposium. Dallas, TX, USA. \\

\end{hangparas}


\subsection*{2015}
\begin{hangparas}{.5in}{1}
Li R, \textbf{Smith DV}, Clithero JA, Venkatraman V, Carter RM, Huettel SA (December, 2015). Revisiting the dual-systems model of choice using fMRI: Cognitive engagement and disengagement explain biases in gain/loss framing. Oral paper (by R Li), De Nederlandse Vereniging voor Psychonomie (The Dutch Association for Psychonomics) Winter Conference.

\textbf{Smith DV}, Wang KS, Delgado MR (October, 2015). The Striatum Multiplexes Distinct Reward Signals. Poster presented at the 45th meeting of the Society for Neuroscience. Chicago, IL, USA.

Cho C, \textbf{Smith DV}, Delgado MR (October, 2015). Individual Differences in Reward Sensitivity Modulate Ventrolateral Prefrontal Cortex Responses to Choice. Poster presented at the 45th meeting of the Society for Neuroscience. Chicago, IL, USA.

\textbf{Smith DV}, Wang KS, Delgado MR (September, 2015). The Striatum Multiplexes Distinct Reward Signals. Poster presented at the 13th meeting of the Society for Neuroeconomics. Miami, FL, USA.

Utevsky AV, \textbf{Smith DV}, Young JS, Huettel SA (June, 2015). Executive Control and Default-Mode Network Connectivity Reflect Effect of Prior Stimulus on Behavior. Poster presented at the 21st meeting of the Organization for Human Brain Mapping. Honolulu, HI, USA.

\textbf{Smith DV}, Clithero JA, Delgado MR, Huettel SA (June, 2015). Parsing Reward: Spatiotemporal Analysis Reveals Distinct Striatal Responses to Reward. Poster/talk presented at the 21st meeting of the Organization for Human Brain Mapping. Honolulu, HI, USA.

Wang KS, \textbf{Smith DV}, Delgado MR (April, 2015). Parsing Affective and Informative Reward Properties in the Striatum: a High-Resolution fMRI Investigation. Poster presented at the 8th meeting of the Social \& Affective Neuroscience Society. Boston, MA, USA.

Lewis AH, \textbf{Smith DV}, Manglani H, Delgado MR (April, 2015). Neural Activation and Functional Connectivity During Extinction Learning with Appetitive and Aversive Conditioned Stimuli. Poster presented at the 8th meeting of the Social \& Affective Neuroscience Society. Boston, MA, USA.

\textbf{Smith DV}, Wang KS, Rigney AE, Delgado MR (March, 2015). Distinct Reward Properties are Encoded via Interactions between Nucleus Accumbens and Temporal Parietal Junction. Poster presented at the 1st meeting of the Scientific Research Network on Decision Neuroscience \& Aging. Miami, FL, USA. \\

\end{hangparas}


\subsection*{2014}
\begin{hangparas}{.5in}{1}
Utevsky A, \textbf{Smith DV}, Venkatraman V, Huettel SA (November, 2014). Breaking apart the social network: Distinct subregions within temporoparietal junction and posterior cingulate cortex track social behavior. Poster presented at the 44th meeting of the Society for Neuroscience. Washington, DC, USA.

\textbf{Smith DV}, Rigney AE, Delgado MR (September, 2014). Distinct Reward Properties are Encoded via Interactions between Ventral Striatum and Dorsolateral Prefrontal Cortex. Poster presented at the 12th meeting of the Society for Neuroeconomics. Miami, FL, USA. \\

\end{hangparas}

\vspace{.5cm}
Please contact me directly for conference presentations prior to 2014. 




%\vspace{1cm}
\vfill{}
%\hrulefill

\begin{center}
{\scriptsize  Last updated: \today\- •\-
% ---- PLEASE LEAVE THIS BACKLINK FOR ATTRIBUTION AS PER CC-LICENSE
Typeset in \href{http://nitens.org/taraborelli/cvtex}{\XeTeX }\\
% ---- FILL IN THE FULL URL TO YOUR CV HERE
\href{https://sites.google.com/a/temple.edu/dvs-lab/SmithDV\_vita.pdf}{https://sites.google.com/a/temple.edu/dvs-lab/SmithDV\_vita.pdf}}
\end{center}

\end{document}