%------------------------------------
% Dario Taraborelli
% Typesetting your academic CV in LaTeX
%
% URL: http://nitens.org/taraborelli/cvtex
% DISCLAIMER: This template is provided for free and without any guarantee 
% that it will correctly compile on your system if you have a non-standard  
% configuration.
% Some rights reserved: http://creativecommons.org/licenses/by-sa/3.0/
%
% Adapted from https://github.com/dartar/cvtex
% by David V. Smith (david.v.smith@temple.edu)
%------------------------------------

%!TEX TS-program = xelatex
%!TEX encoding = UTF-8 Unicode

\documentclass[11pt, letterpaper]{article}
\usepackage{xunicode,fontspec,xltxtra} 
\usepackage{libertine} 

% DOCUMENT LAYOUT
\usepackage{geometry} 
\geometry{letterpaper, textwidth=5.5in, textheight=8.5in, marginparsep=7pt, marginparwidth=.6in}
\setlength\parindent{0in}

% FONTS
\usepackage[usenames,dvipsnames]{color}
\usepackage{xunicode}
\usepackage{xltxtra}
\defaultfontfeatures{Mapping=tex-text}
%\setromanfont [Ligatures={Common}, Numbers={OldStyle}, Variant=01]{Linux Libertine O}
%\setmonofont[Scale=0.8]{Monaco}

% ---- CUSTOM COMMANDS
\chardef\&="E050
\newcommand{\html}[1]{\href{#1}{\scriptsize\textsc{[html]}}}
\newcommand{\pdf}[1]{\href{#1}{\scriptsize\textsc{[pdf]}}}
\newcommand{\doi}[1]{\href{#1}{\scriptsize\textsc{[doi]}}} % need this by every paper so i don't have to worry about the link breaking
\newcommand{\psyarxiv}[1]{\href{#1}{\textit{PsyArXiv}}}
\newcommand{\biorxiv}[1]{\href{#1}{\textit{bioRxiv}}}
\newcommand{\medrxiv}[1]{\href{#1}{\textit{medRxiv}}}

% -- "open science" made explicit 
\newcommand{\neurovault}[1]{\href{#1}{\scriptsize\textsc{[NeuroVault]}}}
\newcommand{\materials}[1]{\href{#1}{\scriptsize\textsc{[Open Materials]}}}
\newcommand{\data}[1]{\href{#1}{\scriptsize\textsc{[Open Data]}}}
\newcommand{\preregistration}[1]{\href{#1}{\scriptsize\textsc{[PreRegistration]}}}


% ---- MARGIN YEARS
\usepackage{marginnote}
\newcommand{\amper{}}{\chardef\amper="E0BD }
\newcommand{\years}[1]{\marginnote{\scriptsize #1}}
\renewcommand*{\raggedleftmarginnote}{}
\setlength{\marginparsep}{7pt}
\reversemarginpar

% HEADINGS
\usepackage{sectsty} 
\usepackage[normalem]{ulem} 
\sectionfont{\mdseries\upshape\LARGE}
\subsectionfont{\mdseries\scshape\large} 
\subsubsectionfont{\mdseries\upshape\large} 

\sectionfont{\sectionrule{0pt}{0pt}{-5pt}{0.8pt}}

%\usepackage{titlesec}
%
%\titleformat{\section}
 % {\normalfont\Large\bfseries}{\thesection}{1em}{}[{\titlerule[0.8pt]}]

%\usepackage{titlesec}
%\titleformat{\section}
% {\normalfont\normalsize\bfseries}{\thesection.}{1em}{}
%\titleformat{\subsection}
%  {\normalfont\normalsize\itshape}{\thesubsection.}{1em}{}
%\titleformat{\subsubsection}
%  {\normalfont\normalsize\itshape}{\thesubsubsection.}{1em}{}

% PDF SETUP
% ---- FILL IN HERE THE DOC TITLE AND AUTHOR
\usepackage[driverfallback=dvipdfm, bookmarks, colorlinks, breaklinks, 
% ---- FILL IN HERE THE TITLE AND AUTHOR
	pdftitle={David V. Smith - vita},
	pdfauthor={David V. Smith},
	pdfproducer={http://nitens.org/taraborelli/cvtex}
]{hyperref}  
\hypersetup{linkcolor=blue,citecolor=blue,filecolor=blue,urlcolor=blue} 


% IMPORT OTHER PACKAGES, INCLUDING PYTHONTEX
\usepackage{amsmath,amssymb}
\usepackage{fullpage}
\usepackage{graphicx}
\usepackage[svgnames]{xcolor}
\usepackage{url}
\urlstyle{same}
\usepackage{academicons}
\usepackage{hanging}
\usepackage{multicol}
\usepackage{enumitem}
\defaultfontfeatures{Extension = .otf}
\usepackage{fontawesome}
\usepackage[makestderr]{pythontex}
\restartpythontexsession{\thesection}
\usepackage[framemethod=TikZ]{mdframed}
\newcommand{\pytex}{Python\TeX}


\usepackage{pgf,interval}



% START DOCUMENT
\begin{document}
\textbf{{\Huge David V. Smith, Ph.D.}}\\[.5cm]
\begin{tabbing}
Assistant Professor \hspace{1.75in} \= \aiGoogleScholar \hspace{.2cm} \= \href{https://scholar.google.com/citations?user=czV7OcQAAAAJ&hl=en}{scholar.google.com} \\ 
Department of Psychology \> \aiResearchGate \> \href{https://www.researchgate.net/profile/David_Smith96/}{researchgate.net} \\
Temple University \> \faGithub \>  \href{https://github.com/DVS-Lab/}{github.com/DVS-Lab} \\
825 Weiss Hall \> \aiOrcid \> \href{https://orcid.org/0000-0001-5754-9633}{orcid.org/0000-0001-5754-9633} \\
1701 North 13th Street \>  \aiPublons \> \href{https://publons.com/author/1204254/david-v-smith}{publons.com} \\
Philadelphia, PA, 19122 \> \aiOSF \> \href{https://osf.io/5zq6h/}{osf.io} \\[.2cm]
phone: 215-204-1552  \> \faTwitter \> \href{https://twitter.com/DVSneuro}{twitter.com/DVSneuro} \\
email: \href{mailto:david.v.smith@temple.edu}{david.v.smith@temple.edu} \> \faUniversity \> \href{https://sites.temple.edu/neuroeconlab/}{sites.temple.edu/neuroeconlab} \\ 
 \end{tabbing}

%\faInstitution
%\faLinux
%\faLink
%\faUniversity
 
%https://osf.io/5zq6h/
%https://twitter.com/DVSneuro
%https://github.com/DVS-Lab
%https://sites.temple.edu/neuroeconlab/
%https://orcid.org/0000-0001-5754-9633
%https://publons.com/author/1204254/david-v-smith
%http://www.researcherid.com/rid/A-5487-2011
%https://www.researchgate.net/profile/David_Smith96


%%\hrule \section*{Professional Experience}
\section*{Professional Experience}
\begin{tabbing}
\years{2017-present} \textsc{Temple University} \hspace{.75in} \= \textbf{Assistant Professor}, Department of Psychology \\[.2cm]
\years{2012-2016} \textsc{Rutgers University} \> \textbf{Postdoctoral Fellow}, Department of Psychology \\[.2cm]
\years{2006-2012} \textsc{Duke University} \> \textbf{Graduate Student}, Center for Cognitive Neuroscience \\
\end{tabbing}


%\hrule \section*{Education \& Training}
\section*{Education and Training}
\begin{tabbing}
\years{2012-2016}\textsc{PostDoc} \hspace{.75in} \= \textsc{Rutgers University} \\
\> Department of Psychology \\
\> \textsc{Advisor}: Dr. Mauricio Delgado \\ [.2cm]

\years{2012}\textsc{Ph.D.} \> \textsc{Duke University} \\
\> Department of Psychology and Neuroscience \\
\> \textsc{Advisor}: Dr. Scott Huettel \\ [.2cm]

\years{2006}\textsc{B.S.} \> \textsc{University of South Carolina} \\
\> Experimental Psychology, \textit{Magna Cum Laude} \\
\> \textsc{Advisors}: Drs. Gordon Baylis \& Chris Rorden \\
 \end{tabbing}


%\hrule \section*{Honors \& Awards}
\section*{Honors and Awards}
\noindent
\years{2021}College of Liberal Arts Research Award, Temple University\\
\years{2019}Excellence in Mentoring, Temple University, Department of Psychology Honors Program\\
\years{2019}Faculty Fellow, Public Policy Lab, Temple University\\
\years{2016}Rising Star, Association for Psychological Science\\
\years{2016}Young Investigator Travel Award, NIDA Symposium on Persistent Maladaptive Behaviors\\
\years{2016}NIDA Director’s Travel Award, The College on Problems of Drug Dependence\\
\years{2015}Ruth L. Kirschstein Postdoctoral National Research Service Award, NIMH\\
\years{2015}Merit Abstract Award, Organization for Human Brain Mapping\\
\years{2015}Travel Award, Scientific Research Network on Decision Neuroscience \& Aging\\
\years{2009}Ruth L. Kirschstein Predoctoral National Research Service Award, NIMH\\
\years{2010}Travel Award, Organization for Human Brain Mapping\\
\years{2009}Fellow, Summer Institute in Cognitive Neuroscience, UC - Santa Barbara\\
\years{2008}Fellow, CIT Flexible Learning Space, Duke University\\
\years{2007}Travel Award, Organization for Human Brain Mapping\\
\years{2006}Roger Black Award for Psychological Research, University of South Carolina\\
\years{2005}\textit{Phi Beta Kappa}, University of South Carolina\\
\years{2005}Fellow, NSF Summer Research Institute, University of South Carolina\\
\years{2005}Stubbs Scholar, University of South Carolina\\
\years{2004}Phi Beta Kappa Freshman Scholar Award, University of South Carolina\\
\years{2004}Baroody Scholar Award, University of South Carolina\\
\years{2003}Abney Scholar, University of South Carolina\\
\years{2002}LIFE Scholarship, University of South Carolina\\


%\hrule \section*{Extramural Funding}
\section*{Grants as Principal Investigator}
[*approximate total costs; †approximate direct costs]
\subsection*{Active}
\years{2017-2021} \textbf{Remote Modulation of Reward Circuits with Noninvasive Brain Stimulation}. (NIH R21-MH113917; *\$420,000). Principal Investigator, with Co-I Krekelberg. \\ [.2cm]
\years{2019-2021} \textbf{Aberrant Reward Sensitivity: Mechanisms Underlying Substance Use}. (NIH R03-DA046733; *\$225,000). Principal Investigator, with Co-Is Alloy, Jarcho, McCloskey, and Chein.

\subsection*{Pending}
\years{2019-2024} \textbf{Parsing Reward with Corticostriatal Network Maps}. (NIH R01-DA048857; *\$2,000,000). Principal Investigator, with Co-Is Alloy, Delgado, and Olino. \\ [.2cm]
\years{2021-2026} \textbf{Social Reward Processing Across the Lifespan: Identifying Risk Factors for Financial Exploitation}. (NIH R01-AG067011; *\$3,419,480). Principal Investigator, with Co-Is Giovannetti, Jarcho, and Olson. \\ [.2cm]
\years{2020-2025} \textbf{Modulating Reward Processing with Transcranial Current Stimulation}. (NIH R01-MH124932; *\$3,250,000). Principal Investigator, with Co-Is Alloy and Krekelberg. 

\subsection*{Completed}
\years{2018-2019} \textbf{Modulating Individual Differences in Reward Sensitivity with Transcranial Current Stimulation}. (Targeted Small Grant Award from the Office of the Vice President for Research at Temple University; †\$10,000). Principal Investigator, with Co-I Reilly. \\ [.2cm]
\years{2017-2018} \textbf{Social Reward and Aging: Identifying the Neural Underpinnings of Peer Influences}. (Pilot Grant from the Scientific Research Network on Decision Neuroscience \& Aging; †\$30,000). Principal Investigator, with Co-Is Fareri, Giovannetti, and Reeck. [Subaward of NIH R24-AG054355 (PI Samanez-Larkin)] \\ [.2cm]
\years{2015-2016} \textbf{Parsing Reward: Identifying Distinct Neural Pathways for Specific Reward Properties}. (NIH F32-MH107175; †\$110,000). Principal Investigator. \\ [.2cm]
\years{2009-2012} \textbf{Neurobiological Underpinnings of Decision Making}. (NIH F31-MH086248; †\$82,000). Principal Investigator.


\section*{Grants as Co-Investigator or Consultant}
[*approximate total costs; †approximate direct costs]
\subsection*{Active}
\years{2018-2021} \textbf{Social Reward Learning in Schizophrenia}. (NIH R21-MH116422; †\$275,000). Consultant, with PI Butler. \\ [.2cm]
\years{2020-2023} \textbf{The role of social support and close relationships on neural and behavioral computations of value}. (NIH R15-MH122927; *\$389,000). Consultant, with PI Fareri. \\ [.2cm]
\years{2019-2021} \textbf{Pupillometry as a Physiological Biomarker for Preclinical Dementia in Minority Aging}. (Alzheimer’s Disease Administrative Supplement for NIH R01-DC013063; *\$388,000). Co-Investigator, with PI Reilly.

\subsection*{Pending}
\years{2020-2025} \textbf{Subtyping psychopathology and neural function in the ABCD project}. (NIH R01-MH124786; *\$1,960,000). Co-Investigator, with PI Olino. \\ [.2cm]
\years{2020-2025} \textbf{The influence of mesolimbic-hippocampal interactions on episodic memory during novelty processing and exploration}. (NIH R01-MH123481; *\$2,425,000). Co-Investigator, with PI Murty.  \\ [.2cm]


% PUBLICATIONS
\section*{Bibliography}
\aiGoogleScholar \hspace{.01cm} \href{https://scholar.google.com/citations?user=czV7OcQAAAAJ&hl=en}{scholar.google.com}: h-index: 24; i10-index: 33; total citations: 2439 \\ 
\lbrack*shared first authorship; †trainee under my supervision; RCR: \textsc{Relative Citation Ratio}\footnote{The \textsc{Relative Citation Ratio} (RCR) is a field- and time-normalized citation metric \href{http://journals.plos.org/plosbiology/article?id=10.1371/journal.pbio.1002541}{(Hutchins et al., 2016, \textit{PLoS Biology})}. NIH-funded papers are the benchmark for RCR: Any paper with RCR = 1.0 has an RCR higher than 50\% of NIH-funded papers. Recent papers and papers that are not indexed on PubMed will not have an RCR. All RCR values were extracted from the \href{https://icite.od.nih.gov/stats}{\textit{iCite} database} using \href{https://github.com/gpoore/pythontex}{\pytex} on \today.}] \\ 

\textit{Where possible, I include links to \href{https://neurovault.org}{NeuroVault} maps, open materials, open data, and preregistrations.}


% Function for extracting RCR
\begin{pycode}
import requests
def get_rcr(pmid):
  response = requests.get("/".join(["https://icite.od.nih.gov/api","pubs",str(pmid)]))
  pub = response.json()
  if pub['relative_citation_ratio'] is not None:
    return round(pub['relative_citation_ratio'],2)
  else:
    return pub['relative_citation_ratio']
\end{pycode}


\subsection*{Preprints Under Review}

\begin{hangparas}{.5in}{1}

Helion C, \textbf{Smith DV}, Jarcho J. When thinking you are better leads to feeling worse: Self-other asymmetries in prosocial behavior and increased anxiety during the Covid-19 pandemic. Preprint available on \medrxiv{https://doi.org/10.1101/2021.02.26.21252547}

Dennison JB†*, Sazhin D†*, \textbf{Smith DV}. Decision Neuroscience: Recent Progress and Ongoing Challenges. Preprint available on \psyarxiv{https://doi.org/10.31234/osf.io/4v2re}. [Invited Advanced Review submission for \textit{Wiley Interdisciplinary Reviews: Cognitive Science}]  \\

\end{hangparas}


\subsection*{Publications}

\begin{hangparas}{.5in}{1}

[45] Tepfer LJ†, Alloy LB, \textbf{Smith DV} (2021, in press). Family History of Depression is Associated with Alterations in Task-Dependent Connectivity between the Cerebellum and Ventromedial Prefrontal Cortex. \textit{Depression and Anxiety}. RCR = \py{get_rcr(33666313)} \doi{https://doi.org/10.1002/da.23143} \neurovault{https://neurovault.org/collections/6130/} \materials{https://osf.io/ju32v/} \preregistration{http://aspredicted.org/blind.php?x=8qw2h3}

[44] Martins D, Rademacher L, Gabay AS, Taylor R, Richey JA, \textbf{Smith DV}, Gorlich KS, Nawijn L, Cremers HR, Wilson R, Bhattacharyya S, Paloyelis Y (2021). Mapping the brain areas underpinning social reward and punishment processing in the human brain: A voxel-based meta-analysis of neuroimaging findings using the Social Incentive Delay task. \textit{Neuroscience \& Biobehavioral Reviews}, 122, 1-17. RCR = \py{get_rcr(33421544)} \doi{https://doi.org/10.1016/j.neubiorev.2020.12.034} \neurovault{https://neurovault.org/collections/7793}

[43] Sazhin D†, Frazier A, Haynes CR†, Johnston C, Chat KY, Dennison JB†, Bart CP, McCloskey M, Chein J, Fareri DS, Alloy LB, Jarcho JM, \textbf{Smith DV} (2020). The Role of Social Reward and Corticostriatal Connectivity in Substance Use. \textit{Journal of Psychiatry and Brain Science}, 5:e200024. RCR = \py{get_rcr(32999307)} \doi{https://doi.org/10.20900/jpbs.20200024} %tmp rcr 

[42] Chen EY, Eickhoff S, Giovannetti T, \textbf{Smith DV} (2020). Reduced gray matter volume in the orbitofrontal cortex is associated with greater body mass index: a coordinate-based meta-analysis. \textit{NeuroImage: Clinical}, 28:102420. RCR = \py{get_rcr(32961404)} \doi{https://doi.org/10.1016/j.nicl.2020.102420} \neurovault{https://neurovault.org/collections/8703/} 

[41] Wang S†*, Tepfer LJ†*, Taren AA*, \textbf{Smith DV} (2020). Functional Parcellation of the Default Mode Network: A Large-Scale Meta-Analysis. \textit{Scientific Reports}, 10:16096. RCR = \py{get_rcr(32999307)} \doi{https://doi.org/10.1038/s41598-020-72317-8} \neurovault{https://identifiers.org/neurovault.collection:6262} \materials{https://zenodo.org/record/3599989\#.Xl2u4y2ZPOQ} 

[40] Botvinik-Nezer R, Holzmeister F, Camerer CF, Dreber A, Huber J, Johannesson M, Kirchler M, Iwanir R, Mumford JA, Adcock A, Avesani P, Baczkowski B, Bajracharya A, Bakst L, Ball S, Barilari M, Bault N, Beaton D, Beitner J, Benoit R, Berkers R, Bhanji J, Biswal B, Bobadilla-Suarez S, Bortolini T, Bottenhorn K, Bowring A, Braem S, Brooks H, Emily Brudner E, Calderon C, Camilleri J, Castrellon J, Cecchetti L, Cieslik E, Cole Z, Collignon O, Cox R, Cunningham W, Czoschke S, Dadi K, Davis C, De Luca A, Delgado MR, Demetriou L, Dennison JB†, Di X, Dickie E, Dobryakova E, Donnat C, Dukart J, Duncan NW, Durnez J, Eed A, Eickhoff S, Erhart A, Fontanesi L, Fricke GM, Galvan A, Gau R, Genon S, Glatard T, Glerean E, Goeman J,  Golowin S, González-García S, Gorgolewski K, Grady C, Green M, Guassi Moreira J, Guest O, Hakimi S, Hamilton JP, Hancock R, Handjaras G, Harry B, Hawco C, Herholz P, Herman G, Heunis S, Hoffstaedter F, Hogeveen J, Holmes S, Hu C, Huettel SA, Hughes M, Iacovella V, Iordan A, Isager P, Isik AI, Jahn A, Johnson M, Johnstone T, Joseph M, Juliano A, Kable J, Kassinopoulos M, Koba C, Kong X, Koscik T, Kucukboyaci NE, Kuhl B, Kupek S, Laird A, Lamm C, Langner R, Lauharatanahirun N, Lee H, Lee S, Leemans A, Leo A, Lesage E, Li F, Li M, Lim PC, Lintz E, Liphardt S, Losecaat Vermeer A, Love B, Mack M, Malpica N, Marins T, Maumet C, McDonald K, McGuire J, Melero H, Méndez Leal A, Meyer B, Meyer K, Mihai P, Mitsis G, Moll J, Nielson D, Nilsonne G, Notter M, Olivetti E, Onicas A, Papale P, Patil K, Peelle JE, Pérez A, Pischedda D, Poline JB, Prystauka Y, Ray S, Reuter-Lorenz P, Reynolds R, Ricciardi E, Rieck J, Rodriguez-Thompson A, Romyn A, Salo T, Samanez-Larkin GR, Sanz-Morales E, Schlichting M, Schultz D, Shen Q, Sheridan M, Shiguang F, Silvers J, Skagerlund K, Smith A, \textbf{Smith DV}, Sokol-Hessner P, Steinkamp S, Tashjian S, Thirion B, Thorp J, Tinghög G, Tisdall L, Tompson S, Toro-Serey C, Torre J, Tozzi L, Truong V, Turella L, van’t Veer AE, Verguts T, Vettel J, Vijayarajah S, Vo K, Wall M, Weeda WD, Weis S, White D, Wisniewski D, Xifra-Porxas A, Yearling E, Yoon S, Yuan R, Yuen K, Zhang L, Zhang X, Zosky J, Nichols TE, Poldrack RA, Schonberg T (2020). Variability in the analysis of a single neuroimaging dataset by many teams. \textit{Nature}, 582, 84-88. RCR = \py{get_rcr(32483374)} \doi{https://doi.org/10.1038/s41586-020-2314-9} \neurovault{https://neurovault.org/collections/6047/} \materials{https://github.com/poldrack/narps} \data{https://openneuro.org/datasets/ds001734/versions/1.0.4} 

[39] Fareri DS, \textbf{Smith DV}, Delgado MR (2020). The role of relationship closeness on network connectivity during trust-based interactions. \textit{Social Cognitive and Affective Neuroscience}, 15(3), 261-271. RCR = \py{get_rcr(32232362)} \doi{https://doi.org/10.1093/scan/nsaa031} 

[38] Wang Y*, Metoki A*, \textbf{Smith DV}, Medaglia JD, Zang Y, Benear S, Popal H, Lin Y, Olson IR (2020). Multimodal Mapping of the Face Connectome. \textit{Nature Human Behaviour}, 4, 397-411. RCR = \py{get_rcr(31988441)} \doi{https://doi.org/10.1038/s41562-019-0811-3}

[37] Butler PD, Hoptman MJ, \textbf{Smith DV}, Ermel JA, Calderone DC, Lee SH, Barch DM (2020). Grant Report on Social Reward Learning in Schizophrenia. \textit{Journal of Psychiatry and Brain Science}, 5:e200004. RCR = \py{get_rcr(32206729)} \doi{https://doi.org/10.20900/jpbs.20200004}

[36] Ng TH†, Alloy LB, \textbf{Smith DV} (2019). Meta-analysis of Reward Processing in Major Depressive Disorder Reveals Distinct Abnormalities within the Reward Circuit. \textit{Translational Psychiatry}, 9(293). RCR = \py{get_rcr(31712555)} \doi{https://doi.org/10.1038/s41398-019-0644-x} \neurovault{https://neurovault.org/collections/3884/} \materials{https://osf.io/sjb4d}

[35] Diehl MM, Lempert K, Parr AC, Ballard I, Steele VR, \textbf{Smith DV} (2018). Toward an Integrative Perspective on the Neural Mechanisms Underlying Persistent Maladaptive Behaviors. \textit{European Journal of Neuroscience}, 48(3), 1870-1883. RCR = \py{get_rcr(30044022)} \doi{https://doi.org/10.1111/ejn.14083}

[34] Utevsky AV, \textbf{Smith DV}, Young JS, Huettel SA (2017). Large-Scale Network Coupling with the Fusiform Cortex Future Social Motivation. \textit{eNeuro}, 4(5): eneuro.0084-17.2017. RCR = \py{get_rcr(29034316)} \doi{https://doi.org/10.1523/ENEURO.0084-17.2017} \neurovault{https://neurovault.org/collections/4804/}

[33] Li R*, \textbf{Smith DV}*, Clithero JA, Venkatraman V, Carter RM, Huettel SA (2017). Reason’s Enemy is Not Emotion: Engagement of Cognitive Control Networks Explain Biases in Gain/Loss Framing. \textit{Journal of Neuroscience}, 37 (13) 3588-3598. RCR = \py{get_rcr(28264981)} \doi{https://doi.org/10.1523/JNEUROSCI.3486-16.2017} \neurovault{http://neurovault.org/collections/1484/}

[32] \textbf{Smith DV} \& Delgado MR (2017). Meta-Analysis of Psychophysiological Interactions: Revisiting Cluster-Level Thresholding and Sample Sizes. \textit{Human Brain Mapping}, 38(1), 588-591. RCR = \py{get_rcr(27543687)} \doi{https://doi.org/10.1002/hbm.23354} \neurovault{https://neurovault.org/collections/1406/}

[31] Cho C, \textbf{Smith DV}, Delgado MR (2016). Reward Sensitivity Enhances Ventrolateral Prefrontal Cortex Activation During Free Choice. \textit{Frontiers in Neuroscience}, 10:529. RCR = \py{get_rcr(27917106)} \doi{https://doi.org/10.3389/fnins.2016.00529} \neurovault{http://neurovault.org/collections/2132}

[30] \textbf{Smith DV}, Gseir M, Speer ME, Delgado MR (2016). Toward a Cumulative Science of Functional Integration: a Meta-Analysis of Psychophysiological Interactions. \textit{Human Brain Mapping}, 37(8), 2904-17. RCR = \py{get_rcr(27145472)} \doi{https://doi.org/10.1002/hbm.23216} \neurovault{https://neurovault.org/collections/1406/}

[29] \textbf{Smith DV}, Rigney AE, Delgado MR (2016). Distinct Reward Properties are Encoded via Corticostriatal Interactions. \textit{Scientific Reports}, 6, 20093. RCR = \py{get_rcr(26831208)} \doi{https://doi.org/10.1038/srep20093} \neurovault{http://neurovault.org/collections/1408}

[28] Bhanji JP, \textbf{Smith DV}, Delgado MR (2016). A Brief Anatomical Sketch of Human Ventromedial Prefrontal Cortex. [Supplementary Note 1 for Delgado et al. (2016). \textit{Nature Neuroscience}, 19(12), 1545-1552]. RCR = None \doi{https://doi.org/10.31234/osf.io/zdt7f} \neurovault{https://neurovault.org/collections/5631/}

[27] Wang KS, \textbf{Smith DV}, Delgado MR (2016). Using fMRI to Study Reward Processing in Humans: Past, Present, and Future. \textit{Journal of Neurophysiology}, 115, 1664-1678. RCR = \py{get_rcr(26740530)} \doi{https://doi.org/10.1152/jn.00333.2015}

[26] \textbf{Smith DV} \& Delgado MR (2015). Reward Processing. In A. W. Toga (Ed.), \textit{Brain Mapping: An Encyclopedic Reference} (1st ed., pp. 361-366). Waltham, MA: Academic Press. Open access postprint on \psyarxiv{https://doi.org/10.31234/osf.io/b3gea}. RCR = None \doi{https://doi.org/10.1016/B978-0-12-397025-1.00255-4}

[25] \textbf{Smith DV}*, Sip KE*, Delgado MR (2015). Functional Connectivity with Distinct Neural Networks Tracks Fluctuations in Gain/Loss Framing Susceptibility. \textit{Human Brain Mapping}, 36(7), 2743-55. RCR = \py{get_rcr(25858445)} \doi{https://doi.org/10.1002/hbm.22804}

[24] Young JS*, \textbf{Smith DV}*, Coutlee CG, Huettel SA (2015). Synchrony Between Sensory and Cognitive Networks is Associated with Subclinical Variation in Autistic Traits. \textit{Frontiers in Human Neuroscience}, 9:146. RCR = \py{get_rcr(25852527)} \doi{https://doi.org/10.3389/fnhum.2015.00146} \neurovault{https://neurovault.org/collections/4805/}

[23] Sip KE, \textbf{Smith DV}, Porcelli AJ, Kar K, Delgado MR (2015). Social Closeness and Feedback Modulate Susceptibility to the Framing Effect. \textit{Social Neuroscience}, 10(1), 35-45. RCR = \py{get_rcr(25074501)} \doi{https://doi.org/10.1080/17470919.2014.944316}

[22] \textbf{Smith DV} \& Delgado MR (2015). Social Nudges: Utility Conferred from Others. \textit{Nature Neuroscience}, 18(6), 791-792. RCR = \py{get_rcr(26007210)} \doi{https://doi.org/10.1038/nn.4031}

[21] Murty VP, Shermohammed M, \textbf{Smith DV}, Carter RM, Huettel SA, Adcock RA (2014). Resting State Networks Distinguish Human Ventral Tegmental Area from Substantia Nigra. \textit{NeuroImage}, 100(1), 580-589. RCR = \py{get_rcr(24979343)} \doi{https://doi.org/10.1016/j.neuroimage.2014.06.047} \neurovault{http://neurovault.org/collections/2485/}

[20] \textbf{Smith DV}, Utevsky AV, Bland AR, Clement NJ, Clithero JA, Harsch AE, Carter RM, Huettel SA (2014). Characterizing Individual Differences in Functional Connectivity Using Dual-Regression and Seed-Based Approaches. \textit{NeuroImage}, 95(1), 1-12. RCR = \py{get_rcr(24662574)} \doi{https://doi.org/10.1016/j.neuroimage.2014.03.042}

[19] \textbf{Smith DV}, Clithero JA, Boltuck SE, Huettel SA (2014). Functional Connectivity with Ventromedial Prefrontal Cortex Reflects Subjective Value for Social Rewards. \textit{Social Cognitive and Affective Neuroscience}, 9(12), 2017-2025. RCR = \py{get_rcr(24493836)} \doi{https://doi.org/10.1093/scan/nsu005}

[18] Utevsky AV, \textbf{Smith DV}, Huettel SA (2014). Precuneus is a Functional Core of the Default-Mode Network. \textit{Journal of Neuroscience}, 34(3), 932-940. RCR = \py{get_rcr(24431451)} \doi{https://doi.org/10.1523/JNEUROSCI.4227-13.2014}

[17] Karnath H-O \& \textbf{Smith DV} (2014). The Next Step in Modern Brain Lesion Analysis: Multivariate Pattern Analysis. \textit{Brain}, 137(9), 2405-2407. RCR = \py{get_rcr(25125587)} \doi{https://doi.org/10.1093/brain/awu180}

[16] Strauman TJ, Detloff AM, Sestokas R, \textbf{Smith DV}, Goetz EL, Rivera C, Kwapil L (2013). What Shall I Be, What Must I Be: Neural Correlates of Personal Goal Activation. \textit{Frontiers in Integrative Neuroscience}, 6:123. RCR = \py{get_rcr(23316145)} \doi{https://doi.org/10.3389/fnint.2012.00123}

[15] \textbf{Smith DV}, Clithero JA, Rorden C, Karnath H-O (2013). Decoding the Anatomical Network of Spatial Attention. \textit{Proceedings of the National Academy of Sciences of the USA}, 110(4), 1518-1523. RCR = \py{get_rcr(23300283)} \doi{https://doi.org/10.1073/pnas.1210126110}

[14] Jelsone-Swain L, \textbf{Smith DV}, Baylis GC (2012). The Effect of Stimulus Duration and Motor Response in Hemispatial Neglect During a Visual Search Task. \textit{PLoS ONE}, 7(5), e37369. RCR = \py{get_rcr(22662149)} \doi{https://doi.org/10.1371/journal.pone.0037369}

[13] Libedinsky C, \textbf{Smith DV}, Teng CS, Namburi P, Chen V, Huettel SA, Chee MLW (2011). Sleep Deprivation Alters Valuation Signals in the Ventromedial Prefrontal Cortex. \textit{Frontiers in Behavioral Neuroscience}, 5:70. RCR = \py{get_rcr(22028686)} \doi{https://doi.org/10.3389/fnbeh.2011.00070}

[12] Clithero JA, Reeck CC, Carter RM, \textbf{Smith DV}, Huettel SA (2011). Nucleus Accumbens Mediates Relative Motivation for Rewards in the Absence of Choice. \textit{Frontiers in Human Neuroscience}, 5:87. RCR = \py{get_rcr(21941472)} \doi{https://doi.org/10.3389/fnhum.2011.00087}

[11] Bland AR, Mushtaq F, \textbf{Smith DV} (2011). Exploiting Trial-to-Trial Variability in Multimodal Experiments. \textit{Frontiers in Human Neuroscience}, 5:80. RCR = \py{get_rcr(21886619)} \doi{https://doi.org/10.3389/fnhum.2011.00080}

[10] Appelbaum LG, \textbf{Smith DV}, Boehler CN, Wen C, Woldorff MG (2011). Rapid Modulation of Sensory Processing Induced by Stimulus Conflict. \textit{Journal of Cognitive Neuroscience}, 23(9), 2620-2628. RCR = \py{get_rcr(20849233)} \doi{https://doi.org/10.1162/jocn.2010.21575 }

[9] Clithero JA, \textbf{Smith DV}, Carter RM, Huettel SA (2011). Within- and Cross-Participant Classifiers Reveal Different Neural Coding of Information. \textit{NeuroImage}, 56(2), 699-708. RCR = \py{get_rcr(20347995)} \doi{https://doi.org/10.1016/j.neuroimage.2010.03.057}

[8] \textbf{Smith DV} \& Huettel SA (2010). Decision Neuroscience: Neuroeconomics. \textit{Wiley Interdisciplinary Reviews: Cognitive Science}, 1(6), 854-871. RCR = \py{get_rcr(22754602)} \doi{https://doi.org/10.1002/wcs.73}

[7] Hayden BY, \textbf{Smith DV}, Platt ML (2010). Cognitive Control Signals in Posterior Cingulate Cortex. \textit{Frontiers in Human Neuroscience}, 4:223. RCR = \py{get_rcr(21160560)} \doi{https://doi.org/10.3389/fnhum.2010.00223}

[6] \textbf{Smith DV}, Davis B, Niu K, Healy E, Bonilha L, Fridriksson J, Morgan P, Rorden C (2010). Spatial Attention Evokes Similar Activation Patterns for Visual and Auditory Stimuli. \textit{Journal of Cognitive Neuroscience}, 22(2), 347-361. RCR = \py{get_rcr(19400684)} \doi{https://doi.org/10.1162/jocn.2009.21241 }

[5] \textbf{Smith DV}, Hayden BY, Truong T-K, Song AW, Platt ML, Huettel SA (2010). Distinct Value Signals in Anterior and Posterior Ventromedial Prefrontal Cortex. \textit{Journal of Neuroscience}, 30(7), 2490-2495. RCR = \py{get_rcr(20164333)} \doi{https://doi.org/10.1523/JNEUROSCI.3319-09.2010}

[4] Hayden BY, \textbf{Smith DV}, Platt ML (2009). Electrophysiological Correlates of Default-Mode Processing in Macaque Posterior Cingulate Cortex. \textit{Proceedings of the National Academy of Sciences of the USA}, 106(14), 5948-5953. RCR = \py{get_rcr(19293382)} \doi{https://doi.org/10.1073/pnas.0812035106}

[3] \textbf{Smith DV} \& Clithero JA (2009). Manipulating Executive Function with Transcranial Direct Current Stimulation. \textit{Frontiers in Integrative Neuroscience}, 3:26. RCR = \py{get_rcr(19847324)} \doi{https://doi.org/10.3389/neuro.07.026.2009}

[2] Clithero JA \& \textbf{Smith DV} (2009). Reference and Preference: How Does the Brain Scale Subjective Value? \textit{Frontiers in Human Neuroscience}, 3:11. RCR = \py{get_rcr(19680434)} \doi{https://doi.org/10.3389/neuro.09.011.2009}

[1] Almor A, \textbf{Smith DV}, Bonilha L, Fridriksson J, Rorden C (2007). What is in a Name? Spatial Brain Circuits are Used to Track Discourse References. \textit{Neuroreport}, 18(12), 1215-1219. RCR = \py{get_rcr(17632270)} \doi{https://doi.org/10.1097/WNR.0b013e32810f2e11} \\

\end{hangparas}


\subsection*{Manuscripts in Preparation or Under Review}

\begin{hangparas}{.5in}{1}


\textbf{Smith DV}, Kragel PA, Clithero JA, Revill KP, Rorden C, Huettel SA, Carter RM (in prep). Medial-Lateral Gradient within the Human Striatum Decodes Social Rewards.

%Lewis AH, \textbf{Smith DV}, Manglani H, Delgado MR (in prep). Neural Activation and Functional Connectivity During Extinction Learning with Appetitive and Aversive Conditioned Stimuli.

Kim ES, Wang KS, \textbf{Smith DV}, Speer ME, Delgado MR (in prep). Neural Correlates of Self-Evaluation Enhancement and Dishonest Decisions.

\textbf{Smith DV}, Wang KS, Delgado MR (in prep). Distinct Spatiotemporal Patterns within the Human Striatum Distinguish Reward and Punishment.

Dobryakova E \& \textbf{Smith DV} (in prep). Reward Enhances Connectivity between the Ventral Striatum and the Default Mode Network. \materials{https://github.com/edobryakova/DobryakovaSmith_HCP} \\

\end{hangparas}


%\hrule \section*{Invited Talks}
\vspace{.4cm}
\section*{Invited Talks and Symposia}
\begin{hangparas}{.5in}{1}

\years{2020} Symposium on ``The Neuroscience of Decision Making in Organizations'' (panelist), Academy of Management in Vancouver, Canada. [canceled due to COVID-19]

\years{2020} Social Decision Making in Older Adults: Risk Factors for Financial Exploitation? Scientific Research Network on Decision Neuroscience and Aging in Kapolei, Hawaii.

\years{2020} Neurocognitive Mechanisms of Social Decision Making in Older Adults. University of Hawaii at Manoa.

\years{2018} Constructing Value: Understanding the Role of Corticostriatal Connectivity. Symposium on Biology of Decision-Making in Paris, France.

\years{2017} Social and Economic Rewards Enhance Connectivity between the Ventral Striatum and the Default Mode Network. Rutgers University---Camden.

\years{2016} Brain Connectivity Shapes Responses to Social and Economic Incentives. The Nathan S. Kline Institute for Psychiatric Research, New York University.

\years{2016} Neural Circuitry Underlying Social and Economic Incentives. Temple University.

\years{2015} Neural Circuitry Underlying Social and Economic Incentives. Bard College.

\years{2015} Characterizing Individual Differences in Brain Connectivity. Kessler Foundation.

\years{2015} Linking Neural Circuits to Social and Economic Incentives: From Valuation to Outcome. Dartmouth College.

\years{2015} Interacting Brain Regions Contribute to a Range of Individual Differences. Kessler Foundation.

\years{2014} Advanced Statistical Procedures in Lesion Analysis: Multivariate Pattern Analysis. Federation of the European Societies of Neuropsychology Summer School in Berlin, Germany.

\years{2014} Characterizing Individual Differences in Decision Making. Sackler Institute for Developmental Psychobiology, Weill Medical College of Cornell University.

\years{2011} Neural Mechanisms of Social Valuation. Rutgers University---Newark.

\years{2011} Neural Mechanisms of Social Valuation. University of South Carolina.

\years{2010} Using FSL for Basic and Advanced Neuroimaging Analyses. Georgia State University / Georgia Tech Center for Advanced Brain Imaging. \\

\end{hangparas}



% \pagebreak
 %\SERVICE}
\vspace{.2cm}
\section*{Service and Professional Activities}

\subsection*{Journal Reviewing \& Editorial Roles}
\years{2018-present}Academic Editor, \textit{PLoS ONE}. \\
\years{2015-2017}Review Editor, \textit{Frontiers in Psychology}, section Decision Neuroscience. \\
\years{2015-2017}Review Editor, \textit{Frontiers in Neuroscience}, section Decision Neuroscience. \\


Ad Hoc Reviewer:
\begin{multicols}{2}
\begin{itemize}[noitemsep]
\itshape
\item Advanced Science
\item Advances in Methods and Practices in Psychological Science
\item Annals of the New York Academy of Sciences
\item Biological Psychiatry
\item Brain
\item Brain and Behavior
\item Brain Structure and Function
\item BMC Neuroscience 
\item Cell Reports
\item Cerebral Cortex 
\item Cognitive, Affective, and Behavioral Neuroscience 
\item Cortex
\item Current Directions in Psychological Science
\item Developmental Neuroscience 
\item Ecological Economics
\item European Journal of Neurology
\item Frontiers in Human Neuroscience 
\item Frontiers in Neuroinformatics 
\item Frontiers in Neurology 
\item Frontiers in Neuroscience 
\item Frontiers in Psychology 
\item Human Brain Mapping 
\item International Journal of Environmental Research and Public Health
\item International Journal of Hyperthermia 
\item Journal of Cognitive Neuroscience 
\item Journal of Neuroscience 
\item Journal of Neuroscience, Psychology, \& Economics
\item Management Information Systems Quarterly 
\item Nature Communications 
\item Nature Human Behaviour 
\item Nature Neuroscience
\item NeuroImage 
\item Neuroimage: Clinical 
\item Neuropsychologia 
\item Neuroscience Research 
\item PLoS Computational Biology 
\item PLoS Biology 
\item PLoS ONE 
\item Proceedings of the National Academy of Sciences of the USA
\item Psychological Medicine 
\item Psychological Science 
\item Psychopathology 
\item Scientific Reports 
\item Social, Cognitive, and Affective Neuroscience 
\item Social Neuroscience
\item WIRES: Cognitive Science
\end{itemize}
\end{multicols}

\aiPublons \hspace{.05cm} \href{https://publons.com/author/1204254/david-v-smith}{publons.com} contains complete reviewing and editing record. \\ [.2cm]
Awards on \textit{Publons}: \\
\years{2018} \href{https://publons.com/awards/2018/esi/?name=David\%20V.\%20Smith&esi=15}{Top Reviewers for Neuroscience \& Behavior} \\
\years{2018} \href{https://publons.com/awards/2018/esi/?name=David\%20V.\%20Smith&esi=22}{Top Reviewers for Multidisciplinary} \\
\years{2017} \href{https://publons.com/awards/institution/?asjc=72&order_by=place}{Top Reviewers for Temple University} \\
\years{2017} \href{https://publons.com/awards/field/?name=David\%20V.\%20Smith\&asjc=100}{Top Reviewers for Neuroscience}


\subsection*{Ad Hoc Grant Reviewing}
\years{2021} German Academic Exchange Service (DAAD), PRIME Program. \\
\years{2020} National Institutes of Health, NPAS Study Section. \\
\years{2019} National Science Foundation, Social Psychology Program. \\
\years{2019} Mind Science Foundation. \\
\years{2018} National Science Foundation, Social Psychology Program. \\
\years{2018} Scientific Research Network on Decision Neuroscience \& Aging, Pilot Grants. \\
\years{2017} FWF Austrian Science Fund, START Program. \\
\years{2017} Swiss National Science Foundation, Humanities and Social Sciences, Division I. \\
\years{2017} Wellcome Trust, Senior Research Fellowship in Basic Biomedical Science. \\
\years{2015} Israel Science Foundation, Individual Research Grant. \\
\years{2014} Scientific Research Network on Decision Neuroscience \& Aging, Pilot Grants.

\subsection*{Departmental Service at Temple University}
\years{2018-present} Co-organizer (with Dr. Mathieu Wimmer), Maladaptive Motivated Behaviors Seminar. \\
\years{2018-2019} Member, Undergraduate Neuroscience Committee. \\
\years{2017-present} Member, Statistics Curriculum Committee. \\ 
\years{2017-2020} Organizer, Neuroimaging Methods Journal Club. \\
\years{2017-2018} Member, Faculty Search Committee in Social / Affective Neuroscience. \\
\years{2017} Member, Subcommittee to evaluate and design deep learning server. \\
\years{2016-2017} Member, Faculty Search Committee in Cognitive / Cognitive Neuroscience.

\subsection*{Society Memberships}
\begin{multicols}{2}
\begin{itemize}[noitemsep]
\item Association for Psychological Science
\item Cognitive Neuroscience Society
\item Eastern Psychological Association
\item New York Academy of Sciences
\item Organization for Human Brain Mapping
\item Social \& Affective Neuroscience Society
\item Society for Neuroeconomics
\item Society for Neuroscience
\item Society for Social Neuroscience
\item Society of Biological Psychiatry
\end{itemize}
\end{multicols}

\subsection*{Other Activities}
\begin{hangparas}{.5in}{1}
Co-Organizer, Duke University Neuroeconomics Journal Club (2008-2009).

Conference Abstract Reviewer: Organization for Human Brain Mapping (2010, 2013-2019, 2021); Cognitive Science Society (2017). \\

\end{hangparas}



% TEACHING
\vspace{.2cm}
\section*{Teaching and Mentoring Activities}
% \subsection*{Instructor of Record at Temple University}
% [\underline{U}ndergraduate, \underline{G}raduate; \underline{S}pring, \underline{F}all; *original course] \\ [.2cm]
% Topics: Brain, Behavior and Cognition (``Decision Making and the Brain") [U]: 2017F
% [*\tiny{shared first authorship}; †\tiny{trainee in Smith Lab}
\subsection*{Teaching Experience}

[\underline{U}ndergraduate, \underline{G}raduate; \underline{S}pring, \underline{F}all; *original course] \\ [.2cm]
Instructor, \textit{Current Topics in Neuroscience} [U], Temple University: 2019F\footnote{Cross-listed with \textit{Topics in Psychology}; Topic: ``Neuroimaging: From Image to Inference"}

Instructor, \textit{Decision Neuroscience}* [U], Temple University: 2019S, 2021S

Instructor, \textit{Foundations of Sensation and Perception} [U], Temple University: 2018F, 2020S, 2021S

Guest Lecturer, \textit{Advanced Neuroanatomy} [G], Lewis Katz School of Medicine, Temple University: 2018S

Instructor, \textit{Topics: Brain, Behavior and Cognition} [U], Temple University: 2017F\footnote{Topic: ``Decision Making and the Brain"}

Guest Lecturer, \textit{The Emotional Brain} [U], Rutgers University: 2014S

Lab Instructor, \textit{Neuroscience Boot Camp} [G], Duke University: 2011F

Teaching Assistant, \textit{Introduction to Cognitive Neuroscience} [U], Duke University: 2010S

Teaching Assistant, \textit{Functional Magnetic Resonance Imaging} [G], Duke University: 2008F

Teaching Assistant, \textit{Brain Waves and Cognition} [U], Duke University: 2008S

Supplemental Instruction Leader, \textit{Psychological Statistics} [U], University of South Carolina: 2006S

Teaching Assistant, \textit{Introductory Psychology} [U], University of South Carolina: 2005F, 2006S \\



\subsection*{Mentoring}

\begin{tabbing}
Graduate students, as primary or co-primary mentor: \\  [.1cm]
\hspace{.5in} \= Jeffrey \= Dennison (2018-present), Ph.D. in Psychology (CNS), expected 2023 \\
\> Xinxu (Karen) Shen (2019-present, co-mentored with Murty), Ph.D. in Psychology (CNS), expected 2024 \\
\> Daniel Sazhin (2019-present), Ph.D. in Psychology (CNS), expected 2024 \\
\> James Wyngaarden (2020-present, co-mentored with Jarcho), Ph.D. in Psychology (Social), expected 2025 \\
\> Yi (Jen) Yang (2020-present, co-mentored with Jarcho), Ph.D. in Psychology (Social), expected 2025 \\
\> Ezgi Kasimoglu (2019-2020), Master's in Neuroscience, \\ 
\> \> Research Assistant at Jefferson. \\
\> Lindsey Tepfer (2017-2020), Master's in Neuroscience \& Research Associate, \\
\> \> Ph.D. student at Dartmouth. \\ [.2cm]

Graduate students (other), as committee member or secondary mentor: \\  [.1cm]
\> Katherine Hackett (2018-present, Clinical Psychology) \\
\> Iris Chat (2018-2020, Clinical Psychology) \\
\> Corinne Bart (2019-2020, Clinical Psychology) \\
\> Dr. Tommy Ng (2017-2020, Clinical Psychology) \\
\> Nicole Henninger (2018-2019, Communications) \\
\> Dr. Sangsuk Yoon (2018, Business School) \\
\> Dr. William Hampton (2017-2018) \\
\> Dr. Ashley Drew (2017) \\
\> Dr. Kylie Alm (2017) \\
\> Dr. Gail Rosenbaum (2017) \\
\> Dr. Trishala Parthasarathi (2016-2017, University of Pennsylvania) \\ [.2cm]

%Undergraduates and post-baccalaureate research assistants: \\
Advisor for undergraduate independent study projects, internships, and research assistantships. Also \\
\> advisor for post-baccalaureate research assistants. Selected examples below: \\  [.1cm]

\> John Marc Cipriaso (Temple, '20), Honors Psychology, \\
\> \> Medical Student at Lewis Katz School of Medicine, Temple University. \\
\> Victoria Kelly (research assistantship, '18-19). \\
\> Brijai Varma (visiting scholar, summer '18, '19), University of Pittsburgh. \\
\> Isaac Levy (visiting scholar, summer '18, spring '20), Oberlin College. \\
\> Benjamin Muzekari (Temple '19), Honors Psychology, \\
\> \> Research Assistant at Duke. \\
\> Jane Gaisinsky (Temple '19), Neuroscience, \\
\> \> Research Assistant at UPenn. \\
\> Shaoming Wang (research assistantship, '17-18), \\
\> \> Ph.D. student at NYU. \\
\> Christian Reice (Temple, '17), Neuroscience, \\
\> \> Research Technician at CHOP.

\end{tabbing}

\begin{hangparas}{.5in}{1}
Prior trainees\footnote{These were individuals who I mentored in neuroimaging analysis while under the supervision of S. Huettel or M. Delgado.} with coauthored publications: Jacob S. Young (Duke undergrad); Sarah Boltuck (Duke undergrad); Amanda Utevsky (Duke grad); Rosa Li (Duke grad); Amy Bland (visiting Duke grad, from U of Manchester); Catherine Cho (Rutgers grad); K. Sally Wang (Rutgers grad); Mouad Gseir (Rutgers undergrad). \\ [.1cm]
\end{hangparas}


\subsection*{Mentored Trainee Awards}

\begin{hangparas}{.5in}{1}

Fatima Umar, Liberal Arts Undergraduate Research Award, for \textit{Meta-Analysis of Striatal Connectivity During Reward Processing} (Spring 2021).

Gemma Goldstein, Liberal Arts Undergraduate Research Award, for \textit{Shared Reward Processing and Substance Use} (Fall 2020). [declined due to COVID-19]

Gemma Goldstein, Liberal Arts Undergraduate Research Award, for \textit{Aberrant Reward Sensitivity: Mechanisms Underlying Substance Use} (Spring 2020). 

Srikar Katta, Liberal Arts Undergraduate Research Award, for \textit{The Neural Mechanisms of Risk Behavior} (Fall 2019).

Benjamin Muzekari, Liberal Arts Undergraduate Research Award, for \textit{Neural Mechanisms Underlying Social Decisions} (Spring 2019).

Benjamin Muzekari, Liberal Arts Undergraduate Research Award, for \textit{Modulating Perceptions of Fairness with Noninvasive Brain Stimulation} (Summer 2018).

Lindsey Tepfer, Career Development Grant, The American Association of University Women (2018-2019).

Jane Gaisinsky, Liberal Arts Undergraduate Research Award, for \textit{Modulating Deception with Noninvasive Brain Stimulation} (Spring 2018).

Jane Gaisinsky, Liberal Arts Undergraduate Research Award, for \textit{Altering Social Interactions with Noninvasive Brain Stimulation} (Fall 2017). \\

\end{hangparas}



% RECENT CONFERENCE PRESENTATIONS
\vspace{.2cm}

\section*{Recent Conference Presentations}

\subsection*{2021}
\begin{hangparas}{.5in}{1}

Sazhin D, Helion C, \textbf{Smith DV} (May, 2021). The Steeper the Curve, the Longer People Delay: Effect of Exponential Information on Decision Making. Poster presented at the Association for Psychological Science. [virtual meeting due to COVID-19]

Chat IK-Y, Bart C, Dennison J, \textbf{Smith DV}, Miller GE, Nusslock R, Alloy LB (May, 2021). Inflammatory Signaling and Corticostriatal Functional Connectivity to Anticipated Valence and Salience of Reward and Threat Stimuli: An Investigation in Depressed vs. Non-Depressed Young Adults. Poster presented at Society for Biological Psychiatry. [virtual meeting due to COVID-19]

Shen X, Meketon EP, \textbf{Smith DV}, Murty VP (March, 2021). Machine learning as an automated approach to scoring free recall of naturalistic stimuli. Poster presented at Cognitive Neuroscience Society. [virtual meeting due to COVID-19]

Hackett K, Katta S, Jarcho J, Fareri DS, Giovannetti T, \textbf{Smith DV} (February, 2021). Relationship between cognition, social support, and susceptibility to fraud among two groups of older adults before and during COVID-19. Poster presented at International Neuropsychology Society. [virtual meeting due to COVID-19]

\end{hangparas}

\subsection*{2020}
\begin{hangparas}{.5in}{1}

Jarcho J, Haynes C, Quarmley M, Johnston C, Smith DV, Cassidy C (December, 2020). Neuromelanin-Sensitive MRI Signal is Associated with Functional Striatal Response to Social but Not Monetary Reward Processing: Potential Mechanisms for Social Anxiety. Annual Meeting of the American College of Neuropsychopharmacology. [virtual meeting due to COVID-19]

Katta S, Hackett K, Jarcho JM, Giovannetti T, Fareri DS, \textbf{Smith DV} (October, 2020). Financial Exploitation in Older Adults: Characterizing the Role of Sociodemographic Factors, Cognition, and Social Decision Making. Poster presented at the 17th meeting of the Society for Neuroeconomics. [virtual meeting due to COVID-19]

Fareri DS, Hackett K, Giovannetti T, \textbf{Smith DV} (October, 2020). Older Adults Exhibit Enhanced Connectivity between Caudate and Default Mode Network During Shared Reward Processing. Poster presented at the 17th meeting of the Society for Neuroeconomics. [virtual meeting due to COVID-19]

Senia N, \textbf{Smith DV}, Fareri DS (March, 2020). Neural responses to drug induced cues in cocaine and heroin users: an activation likelihood meta-analysis. Poster presented at the 27th meeting of the Cognitive Neuroscience Society. Boston, MA, USA. [virtual meeting due to COVID-19]

O'Shea I, \textbf{Smith DV}, Murty V (March, 2020). Differences in Resting-State Midbrain Connectivity in Parkinson’s Disease. Poster presented at the 27th meeting of the Cognitive Neuroscience Society. Boston, MA, USA. [virtual meeting due to COVID-19]

Shen X, Murty V, \textbf{Smith DV} (March, 2020). Actively testing hypothesis using acquired information during encoding enhances delayed memory. Poster presented at the 27th meeting of the Cognitive Neuroscience Society. Boston, MA, USA. [virtual meeting due to COVID-19]

Chen, EY, Giovannetti T, \textbf{Smith DV} (June, 2020). Obesity is associated with reduced orbitofrontal cortex volume: a coordinate-based meta-analysis. Poster presented at the 26th meeting of the Organization for Human Brain Mapping, Montreal, Canada. [virtual meeting due to COVID-19] \\

\end{hangparas}


\subsection*{2019}
\begin{hangparas}{.5in}{1}

Tepfer LJ, Alloy LB, \textbf{Smith DV} (October, 2019). Familial and lifetime history of depression: alterations in the neural circuitry underlying reward and social cognition. Poster presented at the 50th meeting of the Society for Neuroscience. Chicago, IL, USA.

Hackett K, Henninger NM, Kelly V, Giovannetti T, Fareri DS, \textbf{Smith DV} (October, 2019). Response to perceived fairness is associated with reduced connectivity within reward circuitry in older adults. Poster presented at the 50th meeting of the Society for Neuroscience. Chicago, IL, USA.

Kelly V, Hackett K, Henninger NM, Giovannetti T, \textbf{Smith DV}, Fareri DS (October, 2019). Aging alters corticostriatal interactions during shared reward processing. Poster presented at the 50th meeting of the Society for Neuroscience. Chicago, IL, USA.

\textbf{Smith DV}, Liu Y, Krekelberg B (October, 2019). Transcranial alternating current stimulation alters reward-dependent corticostriatal interactions. Poster presented at the 50th meeting of the Society for Neuroscience. Chicago, IL, USA. 

Henninger NM, Kelly V, Hackett K, Fareri DS, Tepfer LJ, Katta S, Reeck C, Giovannetti T, Beard EC, Dennison J, Muzekari B, Desalme DF, Kinmartin R, Lang A, Cipriaso JM, Hunter E, Morrison C, \textbf{Smith DV} (October, 2019). Age‐related reductions in functional connectivity in social brain systems during an economic trust task. Poster presented at the 50th meeting of the Society for Neuroscience. Chicago, IL, USA.

Dennison JB, Ng T, Alloy L, \textbf{Smith DV} (October, 2019). Using Corticostriatal Networks to Disentangle Reward Value and Salience. Poster presented at the 50th meeting of the Society for Neuroscience. Chicago, IL, USA.

\textbf{Smith DV}, Liu Y, Krekelberg B (October, 2019). Transcranial alternating current stimulation alters reward-dependent corticostriatal interactions. Poster presented at the 16th meeting of the Society for Neuroeconomics. Dublin, Ireland. 

Tepfer LJ, Slipenchuk M, Muzekari B, Krekelberg B, \textbf{Smith DV} (June, 2019). Altering Social Norm Compliance with Transcranial Alternating Current Stimulation. Poster presented at the 9th meeting of the Interdisciplinary Symposium on Decision Neuroscience. Durham, North Carolina, USA.

Kelly V, Slipenchuk M, Katta S, Clithero JA, \textbf{Smith DV} (June, 2019). The More the Merrier: Participants Value Having More Options to Choose From. Poster presented at the 9th meeting of the Interdisciplinary Symposium on Decision Neuroscience. Durham, North Carolina, USA.

Henninger NM, Katta S, Kelly V, Hackett K,  Reeck C, Giovannetti T, Fareri DS, \textbf{Smith DV} (June, 2019). Aging is associated with reductions in functional connectivity in social brain systems. Abstract presented at the annual Interdisciplinary Symposium on Decision Neuroscience. Durham, NC, USA.

Fareri DS, Kelly V, Henninger NM, Hackett K, DeSalme D, Muzekari B, Katta S, Reeck C, Giovannetti T, \textbf{Smith DV} (May 2019). The influence of close relationships on shared reward processing in older and younger adults. Poster presented at the 12th meeting of the Social \& Affective Neuroscience Society. Miami, FL, USA.

\textbf{Smith DV}, Henninger NM, Hackett K, Kelly V, DeSalme D, Muzekari B, Katta S, Giovannetti T, Fareri DS (June, 2019). Fairness is Associated with Increased Connectivity between the Executive Control Network and MPFC. Abstract submitted for consideration at the 25th meeting of the Organization for Human Brain Mapping. Rome, Italy.

Ng TH, Alloy LB, \textbf{Smith DV} (March, 2019). Reward Processing in Preadolescents with Bipolar Disorder: An fMRI Study. Abstract submitted for consideration at the 21st meeting of the International Society for Bipolar Disorders. Sydney, Australia.

Tepfer LJ, Slipenchuk M, Muzekari B, Krekelberg B, \textbf{Smith DV} (March, 2019). Altering Social Norm Compliance with Transcranial Alternating Current Stimulation. Poster presented at the 90th meeting of the Eastern Psychological Association. New York, New York, USA.

Muzekari B, Slipenchuk M, Tepfer LJ, Krekelberg B, \textbf{Smith DV} (March, 2019). Modulating Social Influences on Fairness Perception with Transcranial Alternating Current Stimulation. Poster presented at the 90th meeting of the Eastern Psychological Association. New York, New York, USA. \\

\end{hangparas}


\subsection*{2018}
\begin{hangparas}{.5in}{1}

Chiu M, Ng TH, Alloy LB, \textbf{Smith DV} (October, 2018). Linking Valuation Circuitry with Maladaptive Decision Making within the Human Connectome Project. Poster to be presented at the 16th meeting of the Society for Neuroeconomics. Philadelphia, PA, USA.

Chiu M, Ng TH, Alloy LB, \textbf{Smith DV} (May, 2018). Reward-dependent Connectivity with Orbitofrontal Cortex in Subclinical Depression. Poster presented at the 73rd meeting of the Society of Biological Psychiatry. New York, NY, USA.

Zhang H, Venkatraman V, \textbf{Smith DV} (May, 2018). Perceiving Social Interactions Suppresses Connectivity between the Default Mode Network and Ventral Striatum. Poster presented at the 11th meeting of the Social \& Affective Neuroscience Society. New York, NY, USA. \\

\end{hangparas}


\subsection*{2017}
\begin{hangparas}{.5in}{1}
Ng TH, Alloy LB, \textbf{Smith DV} (November, 2017). Reward Processing Abnormalities in Mood Disorders: A Systematic Review and Meta-analysis of Neuroimaging Studies. Poster presented at the 51st meeting of the Association for Behavioral and Cognitive Therapies. San Diego, CA, USA.

Ng TH, Alloy LB, \textbf{Smith DV} (October, 2017). Reward Processing Abnormalities in Unipolar Depression: A Meta-analysis of Neuroimaging Studies. Poster presented at the 15th meeting of the Society for Neuroeconomics. Toronto, ON, Canada.

Wang S, \textbf{Smith DV}, Delgado MR (September, 2017). Informative and Affective Neural Pathways underlying Explore-Exploit Tradeoffs. Poster presented at the inaugural Computational Cognitive Neuroscience conference. New York, NY, USA.

Dobryakova E \& \textbf{Smith DV} (June, 2017). Reward Enhances Connectivity between the Ventral Striatum and the Default Mode Network. Poster presented at the 23rd meeting of the Organization for Human Brain Mapping. Vancouver, BC, Canada.

Wang  S, Taren AA, \textbf{Smith DV} (May, 2017). Large-Scale Meta-Analytic Characterization of the Default Mode Network. Poster presented at the 29th meeting of the Association for Psychological Science. Boston, MA, USA.

Chen EY, Foster GD, Mohamed FB, Conklin CJ, Hoge WS, Olson IR, Chein JM, \textbf{Smith DV}, McCloskey MS, Obradović Z, Olino TM, and the Temple Eating Disorders program represented by Jean M. Arlt (May, 2017). Can Baseline Resting State Functional Connectivity Classify Clinically Significant Weight Loss 3 and 15 Months Later? Poster presented at the 29th meeting of the Association for Psychological Science. Boston, MA, USA.

Chen EY, Olson I, \textbf{Smith DV}, Olino T, McCloskey MS, Chein J, Edwards M, Obradovic Z  (April, 2017). The use of multivariate pattern analysis to develop and test an objective diagnostic clinical test for binge eating disorder in the context of obesity. Poster presented the 3rd meeting of the Association for Clinical and Translational Science. Washington, DC.

Fareri DS, \textbf{Smith DV}, Delgado MR (March, 2017). Reciprocation from a friend enhances coupling between the default mode network and ventral striatum. Talk given (by D. Fareri) at the 10th meeting of the Social \& Affective Neuroscience Society. Los Angeles, CA, USA. \\

\end{hangparas}


\subsection*{2016}
\begin{hangparas}{.5in}{1}
\textbf{Smith DV}, Wang S, Delgado MR (November, 2016). Neural Pathways Underlying Explore-Exploit Tradeoffs in Social and Nonsocial Contexts. Poster presented at the 7th meeting of the Society for Social Neuroscience. San Diego, CA, USA.

\textbf{Smith DV}, Wang S, Delgado MR (November, 2016). Neural Pathways Underlying Explore-Exploit Tradeoffs in Social and Nonsocial Contexts. Poster presented at the 46th meeting of the Society for Neuroscience. San Diego, CA, USA.

Li R, \textbf{Smith DV}, Clithero JA, Venkatraman V, Carter RM, Huettel SA (August, 2016). Revisiting the dual-systems model of choice using fMRI: Cognitive engagement and disengagement explain biases in gain/loss framing. Poster presented at the 14th meeting of the Society for Neuroeconomics. Berlin, Germany.

Utevsky AV, \textbf{Smith DV}, Venkatraman V, Huettel SA (August, 2016). Distinct subregions within the temporoparietal junction and posterior cingulate uniquely track prosocial decision-making. Poster presented at the 14th meeting of the Society for Neuroeconomics. Berlin, Germany.

Hakimi S, Clithero JA, Mullette-Gillman OA, \textbf{Smith DV}, McLaurin E, Taren A, Venkatraman V, Huettel SA, Carter RM (August, 2016). Decomposing Risk Representation in Parietal Cortex. Poster presented at the 14th meeting of the Society for Neuroeconomics. Berlin, Germany.

\textbf{Smith DV}, Li R, Clithero JA, Venkatraman V, Carter RM, Huettel SA (May, 2016). Revisiting the dual-systems model of choice using fMRI: Cognitive engagement and disengagement explain biases in gain/loss framing. Poster presented at the 28th meeting of the Association for Psychological Science. Chicago, IL, USA.

\textbf{Smith DV}, Gseir M, Speer ME, Delgado MR (April, 2016). Toward a Cumulative Science of Functional Integration: a Meta-Analysis of Psychophysiological Interactions. Poster presented at the 9th meeting of the Social \& Affective Neuroscience Society. New York, NY, USA.

\textbf{Smith DV}, Gseir M, Speer ME, Delgado MR (April, 2016). Toward a Cumulative Science of Functional Integration: a Meta-Analysis of Psychophysiological Interactions. Poster presented at the 10th annual Reprogramming the Brain to Health symposium. Dallas, TX, USA. \\

\end{hangparas}


\subsection*{2015}
\begin{hangparas}{.5in}{1}
Li R, \textbf{Smith DV}, Clithero JA, Venkatraman V, Carter RM, Huettel SA (December, 2015). Revisiting the dual-systems model of choice using fMRI: Cognitive engagement and disengagement explain biases in gain/loss framing. Oral paper (by R Li), De Nederlandse Vereniging voor Psychonomie (The Dutch Association for Psychonomics) Winter Conference.

\textbf{Smith DV}, Wang KS, Delgado MR (October, 2015). The Striatum Multiplexes Distinct Reward Signals. Poster presented at the 45th meeting of the Society for Neuroscience. Chicago, IL, USA.

Cho C, \textbf{Smith DV}, Delgado MR (October, 2015). Individual Differences in Reward Sensitivity Modulate Ventrolateral Prefrontal Cortex Responses to Choice. Poster presented at the 45th meeting of the Society for Neuroscience. Chicago, IL, USA.

\textbf{Smith DV}, Wang KS, Delgado MR (September, 2015). The Striatum Multiplexes Distinct Reward Signals. Poster presented at the 13th meeting of the Society for Neuroeconomics. Miami, FL, USA.

Utevsky AV, \textbf{Smith DV}, Young JS, Huettel SA (June, 2015). Executive Control and Default-Mode Network Connectivity Reflect Effect of Prior Stimulus on Behavior. Poster presented at the 21st meeting of the Organization for Human Brain Mapping. Honolulu, HI, USA.

\textbf{Smith DV}, Clithero JA, Delgado MR, Huettel SA (June, 2015). Parsing Reward: Spatiotemporal Analysis Reveals Distinct Striatal Responses to Reward. Poster/talk presented at the 21st meeting of the Organization for Human Brain Mapping. Honolulu, HI, USA.

Wang KS, \textbf{Smith DV}, Delgado MR (April, 2015). Parsing Affective and Informative Reward Properties in the Striatum: a High-Resolution fMRI Investigation. Poster presented at the 8th meeting of the Social \& Affective Neuroscience Society. Boston, MA, USA.

Lewis AH, \textbf{Smith DV}, Manglani H, Delgado MR (April, 2015). Neural Activation and Functional Connectivity During Extinction Learning with Appetitive and Aversive Conditioned Stimuli. Poster presented at the 8th meeting of the Social \& Affective Neuroscience Society. Boston, MA, USA.

\textbf{Smith DV}, Wang KS, Rigney AE, Delgado MR (March, 2015). Distinct Reward Properties are Encoded via Interactions between Nucleus Accumbens and Temporal Parietal Junction. Poster presented at the 1st meeting of the Scientific Research Network on Decision Neuroscience \& Aging. Miami, FL, USA. \\

\end{hangparas}


\subsection*{2014}
\begin{hangparas}{.5in}{1}
Utevsky A, \textbf{Smith DV}, Venkatraman V, Huettel SA (November, 2014). Breaking apart the social network: Distinct subregions within temporoparietal junction and posterior cingulate cortex track social behavior. Poster presented at the 44th meeting of the Society for Neuroscience. Washington, DC, USA.

\textbf{Smith DV}, Rigney AE, Delgado MR (September, 2014). Distinct Reward Properties are Encoded via Interactions between Ventral Striatum and Dorsolateral Prefrontal Cortex. Poster presented at the 12th meeting of the Society for Neuroeconomics. Miami, FL, USA. \\

\end{hangparas}

\vspace{.5cm}
Please contact me directly for conference presentations prior to 2014. 




%\vspace{1cm}
\vfill{}
%\hrulefill

\begin{center}
{\scriptsize  Last updated: \today\- •\-
% ---- PLEASE LEAVE THIS BACKLINK FOR ATTRIBUTION AS PER CC-LICENSE
Typeset in \href{http://nitens.org/taraborelli/cvtex}{\XeTeX }\\
% ---- FILL IN THE FULL URL TO YOUR CV HERE
\href{https://sites.google.com/a/temple.edu/dvs-lab/SmithDV\_vita.pdf}{https://sites.google.com/a/temple.edu/dvs-lab/SmithDV\_vita.pdf}}
\end{center}

\end{document}